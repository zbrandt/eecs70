\begin{homeworkProblem}{Edge Colorings}

    An edge coloring of a graph is an assignment of colors to edges in a graph 
    where any two edges incident to the same vertex have different colors. An 
    example is shown on the left.

    \begin{center}
        \begin{tikzpicture}
            \clip (-1, -1) rectangle (8, 2.1);
    
            \node[circ] (n1) at (0, 0) {};
            \node[circ] (n2) at (1, {sqrt(3)}) {};
            \node[circ] (n3) at (2, 0) {};
            \draw (n1) -- node[above left] {color 1} (n2)
            -- node[above right] {color 2} (n3)
            -- node[below] {color 3} (n1);
    
            \node[circ] (m1) at (5, 0) {};
            \node[circ] (m2) at (5, 2) {};
            \node[circ] (m3) at (7, 2) {};
            \node[circ] (m4) at (7, 0) {};
            \draw (m1) -- (m2) -- (m3) -- (m4) -- (m1);
            \draw (m2) -- (m4);
            \draw (m1) edge[out=-60, in=-30, looseness=2.5] (m3);
        \end{tikzpicture}
    \end{center}

    \begin{itemize}
        \item[A)] Show that the 4 vertex complete graph above can be 3 edge colored.
        (You may use the numbers $1,2,3$ for colors. A figure is shown on the right.)
        \item[B)] Prove that any graph with maximum degree $d \geq 1$ can be edge 
        colored with $2d-1$ colors. 
        \item[C)] Prove that a tree can be edge colored with $d$ colors where $d$
         is the maximum degree of any vertex.
    \end{itemize}

    \part 

    Below is the 4-vertex graph with edges colored such that only 3 colors are used.

    \begin{flushleft}
        \begin{tikzpicture}
            \clip (-1, -1) rectangle (8, 2.1);
    
            \node[circ] (m1) at (5, 0) {};
            \node[circ] (m2) at (5, 2) {};
            \node[circ] (m3) at (7, 2) {};
            \node[circ] (m4) at (7, 0) {};
            \draw[red, thick] (m1) -- (m2);
            \draw[blue, thick] (m2) -- (m3);
            \draw[red, thick] (m3) -- (m4);
            \draw[blue, thick] (m4) -- (m1);
            \draw[brown, thick] (m2) -- (m4);
            \draw[brown, thick] (m1) edge[out=-60, in=-30, looseness=2.5] (m3);
        \end{tikzpicture}
    \end{flushleft}

    \part 
    % Since a graph can be edge colored when any two edges incident to the same 
    % vertex have different colors, any graph with maximum degree $d \geq 1$ can
    % be edge colored with $2d-1$ colors. $2d-1$ is always greater than or equal 
    % to $d$ when $d \geq 1$, and it is always possible to 

    To prove that any graph with maximum degree $d \geq 1$ can be edge colored with
    $2d -1$ colors I will use the principle of induction. 

    \begin{itemize}
        \item \textit{Base case}: When the maximum degree of a graph is $d=1$ 
        there can only be two vertices and one edge. It therefore only takes $2d
        - 1 = 2(1)-1 = 1$ colors to color the edges and the claim holds. 
        \item \textit{Inductive hypothesis}: Assume that any graph with maximum 
        degree $d \geq 1$ can be edge colored with $2d-1$ colors for some $1 \leq d 
        \leq k$ where $k \in \mathbb{N}$
        \item \textit{Inductive step}: For $d = k + 1$, we can show that any graph
        can be edge colored with $2d -1$ colors. There exists at least one vertex
        with degree $k+1$, which we can remove along with its incident edges. The
        remaining graph has a maximum degree of $k$, since the other vertices lost 
        their connection and dropped in degree. By the inductive hypothesis, the 
        remaining graph can be edge colored with $2k-1$ colors. Coloring the graph
        with the removed vertex requires $k+1$ colors (one color for each edge).
        There are $2(k+1)-1=2k+1$ colors at our disposal, which is enough to color
        each edge uniquely. Therefore, the graph can be colored with $2(k+1)-1$
        colors. Therefore, any graph with maximum degree $d \geq 1$ can be edge
        colored with $2d-1$ colors.
    \end{itemize}

    \part 

    The maximum degree of any vertex in a tree must be two, otherwise there would
    be a cycle. Therefore, $d$ is equal to 2. To be edge colored, any two edges 
    incident to the same vertex must have different colors. Since there are two
    colors available, it is possible to edge color the tree. 

    
\end{homeworkProblem}