\begin{homeworkProblem}{Tweaking RSA}

    You are trying to send a message to your friend, and as usual, Eve is trying
    to decipher what the message is. However, you get lazy, so you use $N = p$, 
    and $p$ is prime. Similar to the original method, for any message $x \in 
    \{0,1, \ldots, N-1\}$, $E(x) \equiv x^e$ (mod $N$), and $D(y) \equiv y^d$ 
    (mod $N$).

    \begin{itemize}
        \item[A)] Show how you choose $e,d > 1$ in the encryption and decryption
        function, respectively. Prove the correctness property: the message $x$ 
        is recovered after it goes through your new encryption and decryption 
        functions, $E(x)$ and $D(y)$.
        \item[B)] Can Eve now compute $d$ in the decryption function? If so, by
        what algorithm?
        \item[C)] Now you wonder if you can modify the RSA encryption method to 
        work with three primes ($N = pqr$ where $p, q, r$ are all prime). Explain 
        the modifications made to encryption and decryption and include a proof of 
        correctness showing that $D(E(x)) = x$.
    \end{itemize}
    
    \part 

    For correctness, $D(E(x)) \equiv (x^e)^d \pmod{N}$. But in our case, where 
    $N = p$, and not $N=pq$ as usual, $x \equiv x^{ed} \pmod{p}$. $d$ must be 
    such that $ed \equiv 1 \pmod{p-1}$, and can be expressed as $ed = k(p-1) + 1$.
    Fermat's Little Theorem then shows how $x \equiv x^{ed} \pmod{p}$
    \[
        \begin{split}
            x^{ed} &\equiv x^{k(p-1)+1} \pmod{p} \\
            &\equiv x^{k(p-1)}x \pmod{p} \\
            &\equiv 1 \cdot x \pmod{p} \\
            &\equiv x \pmod{p}
        \end{split}
    \] 

    \part 

    Since $p$ is now known (since $N=p$ and no prime factoring is required) and 
    $e$ is public, Eve can compute $d$ as a multiplicative inverse using Euclid's 
    for
    \[
        ed \equiv 1 \pmod{p-1}
    \]

    \part 

    With three primes, $e$ and $d$ become $ed \equiv 1 \pmod{(p-1)(q-1)(r-1)}$, and
    $x^{ed} \equiv x \pmod{N}$ still needs to hold where $N=pqr$. $ed$ can then be 
    expressed as $ed=k(p-1)(q-1)(r-1)+1$, which then can be used to show the equivalency
    of $x^ed$ and $x$
    \[
        \begin{split}
            x^{ed} &\equiv x^{k(p-1)(q-1)(r-1)+1} \pmod{N} \\
            &\equiv x^{k(p-1)(q-1)(r-1)}x \pmod{N} \\
        \end{split}
    \]
    If a number is divisible by $pqr$, it is also divisible by any of $p$, $q$, 
    and $r$, then if a number is computed modulo $N$, it is equivalent modulo of 
    each of $N$'s prime factors. Then for any of the $(p-1), (q-1),$ and $(r-1)$, 
    we can use Fermat's Little Theorem, for example:
    \[
        \begin{split}
            (x^{p-1})^{k(q-1)(r-1)}-1 \equiv 0 \pmod{p}
        \end{split}
    \] 
    Therefore, $x^{ed} \equiv x \pmod{N}$ for $N=pqr$.

\end{homeworkProblem}