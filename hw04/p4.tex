\begin{homeworkProblem}{Sparsity of Primes}

    A prime power is a number that can be written as $p^i$ for some prime $p$ and some
    positive integer $i$. So, $9 = 3^2$ is a prime power, and so is $8 = 2^3$. $42 = 2 \cdot 3 \cdot 7$ is not
    a prime power.

    Prove that for any positive integer $k$, there exists $k$ consecutive positive integers
    such that none of them are prime powers.

    \emph{Hint: This is a Chinese Remainder Theorem problem. We want to find $n$ such that $(n + 1)$, $(n + 2)$, \ldots, and $(n + k)$ are all not powers of primes. We can enforce this by saying that $n + 1$ through $n + k$ each must have two distinct prime divisors. In your proof, you can choose these prime divisors arbitrarily.}

    \solution

    To find an $n$ such that the consecutive sequence of $(n+1), (n+2), \cdots, (n+k)$ of
    $k$ integers is composed of no prime powers, we need to find a sequence where 
    all the numbers have two distinct prime factors (a prime power, by definition,
    only has one). 

    With two times the numbers of consecutive integers for prime number factors
    we can construct the sequence of equivalencies for Chinese Remainder Theorem
    \[
        \begin{split}
            n + 1 &\equiv 0 \pmod{p_1 p_2} \\
            n + 2 &\equiv 0 \pmod{p_3 p_4} \\
            &\vdots \\
            n + k &\equiv 0 \pmod{p_{2k-1} p_{2k}}
        \end{split}   
    \]

    The Chinese Remainder Thereom states that there is such an $n$ for this setup,
    and we can therefore find a consecutive sequence of non-prime powers.

\end{homeworkProblem}