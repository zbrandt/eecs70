\begin{homeworkProblem}{Euler's Totient Theorem}
    Euler's Totient Theorem states that, if $n$ and $a$ are coprime,
    \[
    a^{\phi(n)} \equiv 1 \pmod{n}
    \]
    where $\phi(n)$ (known as Euler's Totient Function) is the number of positive
    integers less than or equal to $n$ which are coprime to $n$ (including 1). Note that this theorem generalizes Fermat's Little Theorem, since if $n$ is prime, then $\phi(n) = n - 1$. 

    \begin{itemize}
    \item[A)] Let the numbers less than $n$ which are coprime to $n$ be $S = \{ m_1, m_2, \ldots, m_{\phi(n)} \}$. 
    Show that the set
    \[S' = \{am_1 \pmod n, am_2 \pmod n, \ldots, am_{\phi(n)} \pmod n \}\]
    is a permutation of $S$. (Hint: Recall the FLT proof.)

    \item[B)] Prove Euler's Totient Theorem. (Hint: Continue to recall the FLT proof.)
    \item[C)] Note 7 gave two proofs for Theorem 7.1: $$ x^{ed} \equiv x \pmod{N}$$ 
    Use Euler's Totient Theorem to give a third proof of this theorem, for the case that $\gcd(x, N)=1$.
    \end{itemize}

    \part 

    Since $n$ and $a$ are coprime, i.e., $\gcd(n, a) = 1$, the numbers in the set $S'$
    are all distinct. Since none of the elements in the set $S$, and the number of 
    elements in the set is $\phi(n) = n-1$. Therefore, $S'$ is a permutation of $S$, i.e.,
    it includes all the same elements just in a different order. 

    \part 

    Continuing from the proof of Fermat's Last Theorem, if we take the product 
    of all the numbers in $S$
    \[
        1 \cdotp 2 \cdotp \dots \cdotp (n-1) \equiv (n-1)! \pmod{n}
    \]
    But, taking the product of all the numbers in $S'$, the equivalency is
    \[
        a \cdotp 2a \cdotp \dots \cdotp (n-1)a \equiv a^{n-1}(n-1)! \pmod{n}
    \]
    However, since we know that the numbers in $S$ and $S'$ are the same, the
    products of both sets' elements must be the same
    \[
        (n-1)! \equiv a^{n-1}(n-1)! \pmod{n}
    \]
    Multiplying both sides of the equivalency by the multiplicative inverse 
    of $(n-1)!$, the equivalency becomes 
    \[
        a^{n-1} \equiv a^{\phi(n)} \equiv 1 \pmod{n}
    \]

    \part 

    

\end{homeworkProblem}