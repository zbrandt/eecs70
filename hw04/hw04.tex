\documentclass{article}

\usepackage{fancyhdr}
\usepackage{extramarks}
\usepackage{amsmath}
\usepackage{amsthm}
\usepackage{amsfonts}

% TikZ/Graphics
\usepackage{tikz}
\usetikzlibrary{
    automata,  % for markov chains
    arrows.meta,  % arrow tips (technically deprecated)
    backgrounds,  % background layers
    decorations,  % decorations applied to paths
    decorations.markings,  % additional markings on paths
    decorations.pathmorphing,  % deforming decorations
    decorations.pathreplacing,  % replacing paths
    calligraphy,  % for calligraphic brace (must be loaded after decorations)
    patterns,  % patterns for filling areas
    positioning,  % better custom positioning
    shapes,  % various shapes
    intersections,
}

\usepackage{forest}
\usepackage{circuitikz}
\usepackage{pgfplots}
\pgfplotsset{compat=1.18}
\usepgfplotslibrary{
    groupplots,  % grouping pgf plots
}

\usepackage[plain]{algorithm}
\usepackage{algpseudocode}
\usepackage{hyperref}
\usepackage{multicol}
\usepackage{mathrsfs}
\usepackage{pifont}
\usepackage{circledsteps}
\usepackage{xcolor}
\usepackage{parskip}

\usetikzlibrary{automata,positioning}

%
% Basic Document Settings
%

\topmargin=-0.45in
\evensidemargin=0in
\oddsidemargin=0in
\textwidth=6.5in
\textheight=9.0in
\headsep=0.25in

\linespread{1.1}

\pagestyle{fancy}
\lhead{\hmwkAuthorName}
\chead{\hmwkClass\ (\hmwkClassInstructor): \hmwkTitle}
\rhead{\firstxmark}
\lfoot{\lastxmark}
\cfoot{\thepage}

\renewcommand\headrulewidth{0.4pt}
\renewcommand\footrulewidth{0.4pt}

\setlength\parindent{0pt}

%
% Create Problem Sections
%

\newcommand{\enterProblemHeader}[1]{
    \nobreak\extramarks{}{Problem \arabic{#1} continued on next page\ldots}\nobreak{}
    \nobreak\extramarks{Problem \arabic{#1} }{Problem \arabic{#1} continued on next page\ldots}\nobreak{}
}

\newcommand{\exitProblemHeader}[1]{
    \nobreak\extramarks{Problem \arabic{#1} }{Problem \arabic{#1} continued on next page\ldots}\nobreak{}
    \stepcounter{#1}
    \nobreak\extramarks{Problem \arabic{#1}}{}\nobreak{}
}

\setcounter{secnumdepth}{0}
\newcounter{partCounter}
\newcounter{homeworkProblemCounter}
\setcounter{homeworkProblemCounter}{1}
\nobreak\extramarks{Problem \arabic{homeworkProblemCounter}}{}\nobreak{}

%
% Homework Problem Environment
%
% This environment takes an optional argument. When given, it will adjust the
% problem counter. This is useful for when the problems given for your
% assignment aren't sequential. See the last 3 problems of this template for an
% example.
%
\newenvironment{homeworkProblem}[2][-1]{    
    \ifnum#1>0
        \setcounter{homeworkProblemCounter}{#1}
    \fi
    \section{Problem \arabic{homeworkProblemCounter}: #2}
    \setcounter{partCounter}{1}
    \enterProblemHeader{homeworkProblemCounter}
}{
    \exitProblemHeader{homeworkProblemCounter}
}

%
% Homework Details
%   - Title
%   - Due date
%   - Class
%   - Section/Time
%   - Instructor
%   - Author
%

\newcommand{\hmwkTitle}{Homework\ \#4}
\newcommand{\hmwkDueDate}{February 22, 2025}
\newcommand{\hmwkClass}{Discrete Mathematics}
\newcommand{\hmwkClassTime}{Section A}
\newcommand{\hmwkClassInstructor}{Professor Satish Rao}
\newcommand{\hmwkAuthorName}{\textbf{Zachary Brandt}}
\newcommand{\hmwkAuthorEmail}{\href{mailto:zbrandt@berkeley.edu}{zbrandt@berkeley.edu}}

%
% Title Page
%

\title{
    \vspace{2in}
    \textmd{\textbf{\hmwkClass:\ \hmwkTitle}}\\
    \normalsize\vspace{0.1in}\small{Due\ on\ \hmwkDueDate\ at 4:00pm}\\
    \vspace{0.1in}\large{\textit{\hmwkClassInstructor}}
    \vspace{3in}
}

\author{\hmwkAuthorName \\ \hmwkAuthorEmail}
\date{}

\renewcommand{\part}[1]{\textbf{\large Part \Alph{partCounter}}\stepcounter{partCounter}\\}

%
% Various Helper Commands
%

% Useful for algorithms
\newcommand{\alg}[1]{\textsc{\bfseries \footnotesize #1}}

% For derivatives
\newcommand{\deriv}[1]{\frac{\mathrm{d}}{\mathrm{d}x} (#1)}

% For partial derivatives
\newcommand{\pderiv}[2]{\frac{\partial}{\partial #1} (#2)}

% Integral dx
\newcommand{\dx}{\mathrm{d}x}

% Alias for the Solution section header
\newcommand{\solution}{\textbf{\large Solution}}

% Probability commands: Expectation, Variance, Covariance, Bias
\newcommand{\E}{\mathrm{E}}
\newcommand{\Var}{\mathrm{Var}}
\newcommand{\Cov}{\mathrm{Cov}}
\newcommand{\Bias}{\mathrm{Bias}}

\begin{document}

\maketitle

\pagebreak

% \input{hw03/edgecolor}
\begin{homeworkProblem}{Counting on Graphs + Symmetry}
\begin{itemize}

    \item[A)] How many ways are there to color the faces of a cube using exactly 
    $6$ colors, such that each face has a different color? Note: two colorings 
    are considered the same if one can be obtained from the other by rotating 
    the cube in any way.
    
    \solution
    hello

    \item How many ways are there to color a bracelet with $n$ beads using 
    $n$ colors, such that each bead has a different color? Note: two colorings 
    are considered the same if one of them can be obtained by rotating the 
    other.
    
    \solution
    % Solution for Part 2 goes here.

    \item How many distinct undirected graphs are there with $n$ labeled 
    vertices? Assume that there can be at most one edge between any two 
    vertices, and there are no edges from a vertex to itself. The graphs do 
    not have to be connected.
    
    \solution
    % Solution for Part 3 goes here.

    \item How many distinct cycles are there in a complete graph $K_n$ with 
    $n$ vertices? Assume that cycles cannot have duplicated edges. Two cycles 
    are considered the same if they are rotations or inversions of each other 
    (e.g. $(v_1,v_2,v_3,v_1)$, $(v_2,v_3,v_1,v_2)$ and $(v_1,v_3,v_2,v_1)$ 
    all count as the same cycle).
    
    \solution
    % Solution for Part 4 goes here.

\end{itemize}
\end{homeworkProblem}


\pagebreak

% \input{hw03/tourhyper}
\begin{homeworkProblem}{Secret Sharing}
    
    Suppose the Oral Exam questions are created by 2 TAs and 3 Readers. The 
    answers are all encrypted, and we know that:

    \begin{itemize}
        \item[A)] Two TAs together should be able to access the answers
        \item[B)] Three Readers together should be able to access the answers
        \item[C)] One TA and one Reader together should also be able to access the answers
        \item[D)] One TA by themself or two Readers by themselves should not be 
        able to access the answers.
    \end{itemize}

    Design a Secret Sharing scheme to make this work.

    \solution

    Create a polynomial of degree 6, give each TA four points and give each Reader
    three points. If both TAs collaborate they can reconstruct the polynomial to
    find the secret since they have at least 7 points. If all Readers collaborate 
    they can also reconstruct the polynomial to find the secret with 2 points to
    spare. If one TA and one Reader collaborate they can reconstruct the polynomial
    with no points to spare. Since one TA only has four points and two Readers only
    have 6 points (1 point away from being able to reconstruct the polynomial), this
    Secret Sharing scheme satisfies the condition. 

\end{homeworkProblem}

\pagebreak

% \input{hw03/planandcomp}
\begin{homeworkProblem}{Euler's Totient Theorem}
    Euler's Totient Theorem states that, if $n$ and $a$ are coprime,
    \[
    a^{\phi(n)} \equiv 1 \pmod{n}
    \]
    where $\phi(n)$ (known as Euler's Totient Function) is the number of positive
    integers less than or equal to $n$ which are coprime to $n$ (including 1). Note that this theorem generalizes Fermat's Little Theorem, since if $n$ is prime, then $\phi(n) = n - 1$. 

    \begin{itemize}
    \item[A)] Let the numbers less than $n$ which are coprime to $n$ be $S = \{ m_1, m_2, \ldots, m_{\phi(n)} \}$. 
    Show that the set
    \[S' = \{am_1 \pmod n, am_2 \pmod n, \ldots, am_{\phi(n)} \pmod n \}\]
    is a permutation of $S$. (Hint: Recall the FLT proof.)

    \item[B)] Prove Euler's Totient Theorem. (Hint: Continue to recall the FLT proof.)
    \item[C)] Note 7 gave two proofs for Theorem 7.1: $$ x^{ed} \equiv x \pmod{N}$$ 
    Use Euler's Totient Theorem to give a third proof of this theorem, for the case that $\gcd(x, N)=1$.
    \end{itemize}

    \part 

    Since $n$ and $a$ are coprime, i.e., $\gcd(n, a) = 1$, the numbers in the set $S'$
    are all distinct. Since none of the elements in the set $S$, and the number of 
    elements in the set is $\phi(n) = n-1$. Therefore, $S'$ is a permutation of $S$, i.e.,
    it includes all the same elements just in a different order. 

    \part 

    Continuing from the proof of Fermat's Last Theorem, if we take the product 
    of all the numbers in $S$
    \[
        1 \cdotp 2 \cdotp \dots \cdotp (n-1) \equiv (n-1)! \pmod{n}
    \]
    But, taking the product of all the numbers in $S'$, the equivalency is
    \[
        a \cdotp 2a \cdotp \dots \cdotp (n-1)a \equiv a^{n-1}(n-1)! \pmod{n}
    \]
    However, since we know that the numbers in $S$ and $S'$ are the same, the
    products of both sets' elements must be the same
    \[
        (n-1)! \equiv a^{n-1}(n-1)! \pmod{n}
    \]
    Multiplying both sides of the equivalency by the multiplicative inverse 
    of $(n-1)!$, the equivalency becomes 
    \[
        a^{n-1} \equiv a^{\phi(n)} \equiv 1 \pmod{n}
    \]

    \part 

    

\end{homeworkProblem}

\pagebreak

% \input{hw03/modpract}
\begin{homeworkProblem}{Lagrange? More like Lamegrange.}

    \begin{itemize}

        \item[A)] Let's say we wanted to interpolate a polynomial through a 
        single point, $(x_0, y_0)$. What would be the polynomial that we would 
        get? (This is not a trick question. A degree 0 polynomial is fine.)
        
        \item[B)] Call the polynomial from the previous part $f_0(x)$. Now say 
        we wanted to define the polynomial $f_1(x)$ that passes through the 
        points $(x_0, y_0)$ and $(x_1, y_1)$. If we write $f_1(x) = f_0(x) + 
        a_1(x - x_0)$, what value of $a_1$ causes $f_1(x)$ to pass through the 
        desired points?
        
        \item[C)] Now say we want a polynomial $f_2(x)$ that passes through 
        $(x_0, y_0)$, $(x_1, y_1)$, and $(x_2, y_2)$. If we write $f_2(x) = 
        f_1(x) + a_2(x - x_0)(x - x_1)$, what value of $a_2$ gives us the 
        desired polynomial?
        
        \item[D)] Suppose we have a polynomial $f_i(x)$ that passes through the 
        points $(x_0, y_0)$, ..., $(x_i, y_i)$ and we want to find a polynomial 
        $f_{i + 1}(x)$ that passes through all those points and also $(x_{i + 
        1}, y_{i + 1})$. If we define $f_{i + 1}(x) = f_i(x) + a_{i + 1}\prod_{j 
        = 0}^i (x - x_j)$, what value must $a_{i + 1}$ take on?

    \end{itemize}

    \part 

    If we interpolate a polynomial through a single point $(x_0, y_0)$, we get
    a 0 degree polynomial, i.e., the point itself for all $x$: $f_0(x) = y_0$.

    \part 

    When $x=x_0$, $f_1(x)$ is equal to $f_1(x_0)= f_0(x_0) + a_1 (x_0 - x_0)
    = y_0$. Then for $x=x_1$, $f_1(x_1) = f_0(x_1) + a_1 (x_1 - x_0) = y_0 +
    a_1 (x_1 - x_0)$. $f_1(x_1)$ should equal $y_1$, so 
    \[
        \begin{split}
            y_1 &= y_0 + a_1 (x_1 - x_0) \\
            a_1 &= \frac{y_1 - y_0}{x_1 - x_0} = \frac{y_1 - f_0(x_1)}{x_1 - x_0}
        \end{split}
    \]

    \part 

    For $f_2(x)$ evaluated at $x=x_0$ and $x=x_1$, we know that $f_1(x)$ will
    output our desired values of $y_0$ and $y_1$, and since the rest of the 
    terms in $f_2(x)$ evaluate to zero, we know $f_2(x)$ passes through the 
    first two of our points. For $x=x_2$, $f_2(x)$ is $f_2(x_2) = f_1(x_2)
    + a_2 (x_2 -x_0)(x_2-x_1)$. Therefore, $a_2$ must be
    \[
        a_2 = \frac{y_2-f_1(x_2)}{(x_2 - x_0)(x_2-x_1)}
    \]

    \part 

    Like in the previous parts, for values $x_0, x_1, \dots, x_i$, $f_{i+1}(x)$
    will output $y_0, y_1, \dots, y_i$ since the product $\prod_{j=0}^{i}(x-x_j)$
    will end up being zero and $f_{i+1}(x)$ collapses to $f_i(x)$. Like before, 
    consider $f_{i+1}(x_{i+1}) = f_i(x_{i+1}) + a_{i+1} \prod_{j=0}^{i}(x_{i+1}
    - x_j)$ and that it should equal $y_{i+1}$. Therefore, $a_{i+1}$ is
    \[
        \begin{split}
            y_{i+1} &= f_i(x_{i+1}) + a_{i+1} \prod_{j=0}^{i} (x_{i+1} - x_j) \\
            a_{i+1} &= \frac{y_{i+1} - f_{i}(x_{i+1})}{\prod_{j=0}^{i}(x_{i+1}-x_j)}
        \end{split}
    \]


\end{homeworkProblem}


\pagebreak

% \input{hw03/wilsonsol}
\begin{homeworkProblem}{RSA Practice}

    Consider the following RSA scheme and answer the specified questions.
    \begin{itemize}
        \item[A)] Assume for an RSA scheme we pick $2$ primes $p = 5$ and $q = 11$
        with encryption key $e = 9$, what is the decryption key $d$? Calculate the 
        exact value.
        \item[B)] If the receiver gets 4, what was the original message?
        \item[C)] Encrypt your answer from B) to check its correctness.        
    \end{itemize}

    \part 

    The number $d$ is the multiplicative inverse of $e$ modulos $(p-1)(q-1)$,
    that is, $d \equiv e^{-1} \pmod{(p-1)(q-1)}$. In our case, $d$ becomes
    \[
        \begin{split}
            d &\equiv e^{-1} \pmod{(p-1)(q-1)} \\
            &\equiv 9^{-1} \pmod{(5-1)(11-1)} \\
            &\equiv 9 \pmod{40}
        \end{split}
    \]
    
    \part 

    To decrypt the message $y=E(x)$, we must compute $D(y) \equiv y^d \pmod{N}$
    \[
        \begin{split}
            x &\equiv D(y) \pmod{N} \\
            &\equiv y^d \pmod{N} \\
            &\equiv 4^9 \pmod{55} \\
            &\equiv 4^3 \cdot 4^3 \cdot 4^3 \pmod{55} \\
            &\equiv 9 \cdot 9 \cdot 9 \pmod{55} \\
            &\equiv 26 \cdot 9 \pmod{55} \\
            &\equiv 14 \pmod{55} 
        \end{split}
    \]

    \part 

    From the last part, $x=14$, to encrypt we need to compute $y=E(x) \equiv x^e \pmod{N}$\
    \[
        \begin{split}
            y & \equiv E(x) \pmod{N} \\
            &\equiv x^e \pmod{N} \\
            &\equiv 14^9 \pmod{55} \\
            &\equiv (14^2)^4 \cdot 14 \pmod{55} \\
            &\equiv (31^2)^2 \cdot 14 \pmod{55} \\
            &\equiv (26)^2 \cdot 14 \pmod{55} \\
            &\equiv 16 \cdot 14 \pmod{55} \\
            &\equiv 4 \pmod{55}
        \end{split}
    \]
    
    The encrypted message from part B) decrypted resulted again in 4.

\end{homeworkProblem}

\pagebreak

% \input{hw03/howmany}
\begin{homeworkProblem}{Tweaking RSA}

    You are trying to send a message to your friend, and as usual, Eve is trying
    to decipher what the message is. However, you get lazy, so you use $N = p$, 
    and $p$ is prime. Similar to the original method, for any message $x \in 
    \{0,1, \ldots, N-1\}$, $E(x) \equiv x^e$ (mod $N$), and $D(y) \equiv y^d$ 
    (mod $N$).

    \begin{itemize}
        \item[A)] Show how you choose $e,d > 1$ in the encryption and decryption
        function, respectively. Prove the correctness property: the message $x$ 
        is recovered after it goes through your new encryption and decryption 
        functions, $E(x)$ and $D(y)$.
        \item[B)] Can Eve now compute $d$ in the decryption function? If so, by
        what algorithm?
        \item[C)] Now you wonder if you can modify the RSA encryption method to 
        work with three primes ($N = pqr$ where $p, q, r$ are all prime). Explain 
        the modifications made to encryption and decryption and include a proof of 
        correctness showing that $D(E(x)) = x$.
    \end{itemize}
    
    \part 

    For correctness, $D(E(x)) \equiv (x^e)^d \pmod{N}$. But in our case, where 
    $N = p$, and not $N=pq$ as usual, $x \equiv x^{ed} \pmod{p}$. $d$ must be 
    such that $ed \equiv 1 \pmod{p-1}$, and can be expressed as $ed = k(p-1) + 1$.
    Fermat's Little Theorem then shows how $x \equiv x^{ed} \pmod{p}$
    \[
        \begin{split}
            x^{ed} &\equiv x^{k(p-1)+1} \pmod{p} \\
            &\equiv x^{k(p-1)}x \pmod{p} \\
            &\equiv 1 \cdot x \pmod{p} \\
            &\equiv x \pmod{p}
        \end{split}
    \] 

    \part 

    Since $p$ is now known (since $N=p$ and no prime factoring is required) and 
    $e$ is public, Eve can compute $d$ as a multiplicative inverse using Euclid's 
    for
    \[
        ed \equiv 1 \pmod{p-1}
    \]

    \part 

    With three primes, $e$ and $d$ become $ed \equiv 1 \pmod{(p-1)(q-1)(r-1)}$, and
    $x^{ed} \equiv x \pmod{N}$ still needs to hold where $N=pqr$. $ed$ can then be 
    expressed as $ed=k(p-1)(q-1)(r-1)+1$, which then can be used to show the equivalency
    of $x^ed$ and $x$
    \[
        \begin{split}
            x^{ed} &\equiv x^{k(p-1)(q-1)(r-1)+1} \pmod{N} \\
            &\equiv x^{k(p-1)(q-1)(r-1)}x \pmod{N} \\
        \end{split}
    \]
    If a number is divisible by $pqr$, it is also divisible by any of $p$, $q$, 
    and $r$, then if a number is computed modulo $N$, it is equivalent modulo of 
    each of $N$'s prime factors. Then for any of the $(p-1), (q-1),$ and $(r-1)$, 
    we can use Fermat's Little Theorem, for example:
    \[
        \begin{split}
            (x^{p-1})^{k(q-1)(r-1)}-1 \equiv 0 \pmod{p}
        \end{split}
    \] 
    Therefore, $x^{ed} \equiv x \pmod{N}$ for $N=pqr$.

\end{homeworkProblem}

\end{document}