\documentclass[11pt]{article}
\usepackage{header}
\def\title{HW 10}

\begin{document}
\maketitle
\fontsize{12}{15}\selectfont

\begin{center}
    Due: Saturday, 4/5, 4:00 PM \\
    Grace period until Saturday, 4/5, 6:00 PM \\
\end{center}

\section*{Sundry}
Before you start writing your final homework submission, state briefly how you 
worked on it.  Who else did you work with?  List names and email addresses. (In
case of homework party, you can just describe the group.)

\begin{center}
    \textcolor{blue}{
        Zachary Brandt \\
        \nolinkurl{zbrandt@berkeley.edu}
    }
\end{center}

\vspace{15pt}

\Question{Probability Potpourri}

\notelinks{\href{https://www.eecs70.org/assets/pdf/notes/n13.pdf}{Note 13},
\href{https://www.eecs70.org/assets/pdf/notes/n14.pdf}{Note 14}}
Provide brief justification for each part.

\begin{Parts}

\Part For two events $A$ and $B$ in any probability space, show that $\Pr[A 
\setminus B] \geq \Pr[A] - \Pr[B]$.

\Part Suppose $\Pr[D \mid C] = \Pr[D \mid \overline{C}]$, where $\overline{C}$ 
is the complement of $C$. Prove that $D$ is independent of $C$.

\Part  If $A$ and $B$ are disjoint, does that imply they're independent?

\end{Parts}

\begin{solution}

\begin{Parts}

\Part $\Pr[A \setminus B]$ is the probability of event $A$ occuring and event 
$B$ not occuring, i.e., $\Pr[A \setminus B] = \Pr[A] - \Pr[A \cap B]$. This is 
greater than or equal to $\Pr[A] - \Pr[B]$, because $\Pr[A \cap B]$ can
only ever be as great as $\Pr[B]$ (or $\Pr[A]$), considering that it is
the intersection of events $A$ and $B$. When $A$ perfectly coinsides with $B$, 
then $\Pr[A \cap B] = \Pr[B]$, but it is otherwise less than $\Pr[B]$.
\[
    \begin{split}
        \Pr[A \cap B] &\leq \Pr[B] \\
        -\Pr[B] &\leq -\Pr[A \cap B] \\
        \Pr[A] - \Pr[B] &\leq \Pr[A] -\Pr[A \cap B] \\
        \Pr[A] - \Pr[B] &\leq \Pr[A \setminus B] 
    \end{split}
\]

\Part For $D$ to be independent of $C$, it must be the case that $\Pr[D \mid C]
= \Pr[D]$. 

\[
    \begin{split}
        \Pr[D \mid C] &= \Pr[D \mid \overline{C}] \\
        \Pr[D \mid C] &= \frac{\Pr[D \cap \overline{C}]}{\Pr[\overline{C}]} \\
        \Pr[D \mid C] &= \frac{\Pr[D \cap \overline{C}]}{1-\Pr[C]} \\
        \Pr[D \mid C](1-\Pr[C]) &= \Pr[D \cap \overline{C}] \\
        \Pr[D \mid C]-\Pr[D \mid C] \cdot \Pr[C] &= \Pr[D \cap \overline{C}] \\
        \Pr[D \mid C]-\Pr[D \cap C] &= \Pr[D \cap \overline{C}] \\
        \Pr[D \mid C] &= \Pr[D \cap \overline{C}] + \Pr[D \cap C]\\
        \Pr[D \mid C] &= \Pr[D]
    \end{split}
\]

Therefore, $D$ is independent of $C$. 

\Part If $A$ and $B$ are disjoint, then knowing that one event happens provides
information on the other. $\Pr[A \cap B] = 0$ if disjoint, and therefore, 
$\Pr[A \mid B] = 0 \neq A$. This does not imply they are independent of each
other. 

\end{Parts}

\end{solution}


\Question{Independent Complements}

\notelinks{\href{https://www.eecs70.org/assets/pdf/notes/n14.pdf}{Note 14}}
Let $\Omega$ be a sample space, and let $A,B \subseteq \Omega$ be two independent events.

\begin{Parts}

\Part Prove or disprove: $\overline{A}$ and $\overline{B}$ must be independent.

\Part Prove or disprove: $A$ and $\overline{B}$ must be independent.

\Part Prove or disprove: $A$ and $\overline{A}$ must be independent.

\Part Prove or disprove: It is possible that $A=B$.

\end{Parts}

\Question{Cliques in Random Graphs}

\notelinks{\href{https://www.eecs70.org/assets/pdf/notes/n13.pdf}{Note 13},\href{https://www.eecs70.org/assets/pdf/notes/n14.pdf}{Note 14}}
Consider the graph $G = (V,E)$ on $n$ vertices which is generated by the following random process: for each pair of vertices $u$ and $v$, we flip a fair coin and place an (undirected) edge between $u$ and $v$ if and only if the coin comes up heads.

\begin{Parts}
\Part What is the size of the sample space?

\Part A $k$-clique in a graph is a set $S$ of $k$ vertices which are pairwise adjacent (every pair of vertices is connected by an edge). For example, a $3$-clique is a triangle. Let $E_S$ be the event that a set $S$ forms a clique. What is the probability of $E_S$ for a particular set $S$ of $k$ vertices? 

\Part Suppose that $V_1 = \{v_1, \dots, v_{\ell}\}$ and $V_2 = \{w_1, \dots, w_k\}$ are two arbitrary sets of vertices. What conditions must $V_1$ and $V_2$ satisfy in order for $E_{V_1}$ and $E_{V_2}$ to be independent? Prove your answer.

\Part Prove that $\binom{n}{k} \le n^k$. (You might find this useful in part (e)).

\Part Prove that the probability that the graph contains a $k$-clique, for $k \geq 4{\log_2 n}+1$, is at most $1/n$. \textit{Hint:} Use the union bound.
\end{Parts}

\Question{Poisoned Smarties}

\notelinks{\href{https://www.eecs70.org/assets/pdf/notes/n14.pdf}{Note 14}}
Supposed there are 3 people who are all owners of their own Smarties factories. Burr Kelly, being the brightest and most innovative of the owners, produces considerably more Smarties than her competitors and has a commanding 50\% of the market share. Yousef See, who inherited her riches, lags behind Burr and produces 40\% of the world's Smarties. Finally Stan Furd, brings up the rear with a measly 10\%. However, a recent string of Smarties related food poisoning has forced the FDA investigate these factories to find the root of the problem. Through her investigations, the inspector found that 2 Smarties out of every 100 at Kelly's factory was poisonous. At See's factory, 5\% of Smarties produced were poisonous. And at Furd's factory, the probability a Smarty was poisonous was 0.1. 

\begin{Parts}
\Part
What is the probability that a randomly selected Smarty will be safe to eat?

\Part
If we know that a certain Smarty didn't come from Burr Kelly's factory, what is the probability that this Smarty is poisonous?

\Part
If a randomly selected Smarty is poisonous, what is the probability it came from Stan Furd's Smarties Factory?

\end{Parts}

\Question{Symmetric Marbles}

\notelinks{\href{https://www.eecs70.org/assets/pdf/notes/n14.pdf}{Note 14}}
A bag contains 4 red marbles and 4 blue marbles. Rachel and Brooke play a game where they draw four marbles in total, one by one, uniformly at random, without replacement. Rachel wins if there are more red than blue marbles, and Brooke wins if there are more blue than red marbles. If there are an equal number of marbles, the game is tied.
\begin{Parts}
    \Part Let $A_1$ be the event that the first marble is red and let $A_2$ be the event that the second marble is red. Are $A_1$ and $A_2$ independent?
    
    \Part What is the probability that Rachel wins the game?
    
    \Part Given that Rachel wins the game, what is the probability that all of the marbles were red?
\end{Parts}
Now, suppose the bag contains 8 red marbles and 4 blue marbles and we add a tiebreaker to the game: if there are an equal number of red and blue marbles among the four drawn, Rachel wins if the third marble is red, and Brooke wins if the third marble is blue. 
\begin{Parts}[resume]
    \Part What is the probability that the third marble is red?
    
    \Part Given that there are $k$ red marbles among the four drawn, where $0 \leq k \leq 4$, what is the probability that the third marble is red? Answer in terms of $k$.
    
    \Part Given that the third marble is red, what is the probability that Rachel wins the game?
    
\end{Parts}

\Question{Socks}
\notelinks{\href{https://www.eecs70.org/assets/pdf/notes/n13.pdf}{Note 13},\href{https://www.eecs70.org/assets/pdf/notes/n14.pdf}{Note 14}}
Suppose you have $n$ different pairs of socks ($n$ left socks and $n$ right socks, for $2n$ individual socks total) in your dresser. 
You take the socks out of the dresser one by one without looking and lay them out in a row on the floor. 
In this question, we'll go through the computation of the probability that no two matching socks are next to each other.

\begin{Parts}
    \Part We can consider the sample space as the set of length $2n$ permutations. What is the size of the sample space $\Omega$, and what is the probability of a particular permutation $\omega \in \Omega$?

    \Part Let $A_i$ be the event that the $i$th pair of matching socks are next to each other. Calculate $\Pr[A_i]$.
    \Part Calculate $\Pr[A_1 \cap  ... \cap A_k]$ for an arbitrary $k \geq 2$. (Hint: try using a counting based approach.)
    \Part Putting these all together, calculate the probability that there is at least one pair of matching socks next to each other. Your answer can (and should) be expressed as a summation. (Hint: use Inclusion/Exclusion.)
    \Part Using your answer from the previous part, what is the probability that no two matching socks are next to each other? (This should follow directly from your answer to the previous part, and also can be left as a summation.)
\end{Parts}

\end{document}
