\documentclass[11pt]{article}
\usepackage{header}
\def\title{HW 10}

\begin{document}
\maketitle
\fontsize{12}{15}\selectfont

\begin{center}
    Due: Saturday, 4/5, 4:00 PM \\
    Grace period until Saturday, 4/5, 6:00 PM \\
\end{center}

\section*{Sundry}
Before you start writing your final homework submission, state briefly how you 
worked on it.  Who else did you work with?  List names and email addresses. (In
case of homework party, you can just describe the group.)

\begin{center}
    \textcolor{blue}{
        Zachary Brandt \\
        \nolinkurl{zbrandt@berkeley.edu}
    }
\end{center}

\vspace{15pt}

\Question{Probability Potpourri}

\notelinks{\href{https://www.eecs70.org/assets/pdf/notes/n13.pdf}{Note 13},
\href{https://www.eecs70.org/assets/pdf/notes/n14.pdf}{Note 14}}
Provide brief justification for each part.

\begin{Parts}

\Part For two events $A$ and $B$ in any probability space, show that $\Pr[A 
\setminus B] \geq \Pr[A] - \Pr[B]$.

\Part Suppose $\Pr[D \mid C] = \Pr[D \mid \overline{C}]$, where $\overline{C}$ 
is the complement of $C$. Prove that $D$ is independent of $C$.

\Part  If $A$ and $B$ are disjoint, does that imply they're independent?

\end{Parts}

\begin{solution}

\begin{Parts}

\Part $\Pr[A \setminus B]$ is the probability of event $A$ occuring and event 
$B$ not occuring, i.e., $\Pr[A \setminus B] = \Pr[A] - \Pr[A \cap B]$. This is 
greater than or equal to $\Pr[A] - \Pr[B]$, because $\Pr[A \cap B]$ can
only ever be as great as $\Pr[B]$ (or $\Pr[A]$), considering that it is
the intersection of events $A$ and $B$. When $A$ perfectly coinsides with $B$, 
then $\Pr[A \cap B] = \Pr[B]$, but it is otherwise less than $\Pr[B]$.
\[
    \begin{split}
        \Pr[A \cap B] &\leq \Pr[B] \\
        -\Pr[B] &\leq -\Pr[A \cap B] \\
        \Pr[A] - \Pr[B] &\leq \Pr[A] -\Pr[A \cap B] \\
        \Pr[A] - \Pr[B] &\leq \Pr[A \setminus B] 
    \end{split}
\]

\Part For $D$ to be independent of $C$, it must be the case that $\Pr[D \mid C]
= \Pr[D]$. 

\[
    \begin{split}
        \Pr[D \mid C] &= \Pr[D \mid \overline{C}] \\
        \Pr[D \mid C] &= \frac{\Pr[D \cap \overline{C}]}{\Pr[\overline{C}]} \\
        \Pr[D \mid C] &= \frac{\Pr[D \cap \overline{C}]}{1-\Pr[C]} \\
        \Pr[D \mid C](1-\Pr[C]) &= \Pr[D \cap \overline{C}] \\
        \Pr[D \mid C]-\Pr[D \mid C] \cdot \Pr[C] &= \Pr[D \cap \overline{C}] \\
        \Pr[D \mid C]-\Pr[D \cap C] &= \Pr[D \cap \overline{C}] \\
        \Pr[D \mid C] &= \Pr[D \cap \overline{C}] + \Pr[D \cap C]\\
        \Pr[D \mid C] &= \Pr[D]
    \end{split}
\]

Therefore, $D$ is independent of $C$. 

\Part If $A$ and $B$ are disjoint, then knowing that one event happens provides
information on the other. $\Pr[A \cap B] = 0$ if disjoint, and therefore, 
$\Pr[A \mid B] = 0 \neq \Pr[A]$. This does not imply they are independent of 
each other. 

\end{Parts}

\end{solution}


\Question{Independent Complements}

\notelinks{\href{https://www.eecs70.org/assets/pdf/notes/n14.pdf}{Note 14}}
Let $\Omega$ be a sample space, and let $A,B \subseteq \Omega$ be two independent '
events.

\begin{Parts}

\Part Prove or disprove: $\overline{A}$ and $\overline{B}$ must be independent.

\Part Prove or disprove: $A$ and $\overline{B}$ must be independent.

\Part Prove or disprove: $A$ and $\overline{A}$ must be independent.

\Part Prove or disprove: It is possible that $A=B$.

\end{Parts}

\begin{solution}

\begin{Parts}
    
\Part True: If $\overline{A}$ and $\overline{B}$ are independent, then 
$\Pr[\overline{A} \cap \overline{B}] = \Pr[\overline{A}] \cdot \Pr[\overline{B}]$.
Also, from observation of a Venn diagram, $\overline{A} \cap \overline{B} = 1 -
A \cup B$. Then, 
\[
    \begin{split}
        \Pr[\overline{A}] \cdot \Pr[\overline{B}] &= (1-\Pr[A]) \cdot (1-\Pr[B]) \\
        &= 1 - \Pr[B] - \Pr[A] + \Pr[A] \cdot \Pr[B] \\
        \text{Since $A$ and $B$ are independent} \quad &= 
        1 - \Pr[B] - \Pr[A] + \Pr[A \cap B] \\
        &= 1 - (\Pr[B] + \Pr[A] - \Pr[A \cap B]) \\ 
        &= 1 - \Pr[A \cup B] \\ 
        &= \Pr[\overline{A} \cap \overline{B}]
    \end{split}
\]

\Part True: If $A$ and $\overline{B}$ are independent, then 
$\Pr[A \cap \overline{B}] = \Pr[A] \cdot \Pr[\overline{B}]$. Also, from
observation, like of a Venn Diagram, $A - A \cap B = A \cap \overline{B}$. Then,
\[
    \begin{split}
        \Pr[A] \cdot \Pr[\overline{B}] &= \Pr[A] \cdot (1 - \Pr[B]) \\
        &= \Pr[A] - \Pr[A] \cdot \Pr[B] \\
        \text{Since $A$ and $B$ are independent} \quad
        &= \Pr[A] - \Pr[A \cap B] \\
        &= \Pr[A \cup \overline{B}]
    \end{split}
\]

\Part False: For $A$ and $\overline{A}$ to be independent, $\Pr[A \mid \overline{A}]
= \Pr[A \cap \overline{A}] / \Pr[\overline{A}] = \Pr[A]$. But since $\Pr[A \cap 
\overline{A}]$ is equal to zero, $\Pr[A \mid \overline{A}] \neq 0$, and the
events are not independent. 

\Part True: it is possible for $A$ and $B$ to be independent when $A=B$. Consider
the case where $\Pr[A] = 1$ (and $\Pr[B] = 1$ then too). Then $\Pr[A \mid B] = 
\Pr[A \cap B] \div \Pr[B] = 1 \div 1 = 1 = \Pr[A]$. 

\end{Parts}

\end{solution}

\Question{Cliques in Random Graphs}

\notelinks{\href{https://www.eecs70.org/assets/pdf/notes/n13.pdf}{Note 13},
\href{https://www.eecs70.org/assets/pdf/notes/n14.pdf}{Note 14}}
Consider the graph $G = (V,E)$ on $n$ vertices which is generated by the following 
random process: for each pair of vertices $u$ and $v$, we flip a fair coin and 
place an (undirected) edge between $u$ and $v$ if and only if the coin comes up 
heads.

\begin{Parts}
\Part What is the size of the sample space?

\Part A $k$-clique in a graph is a set $S$ of $k$ vertices which are pairwise 
adjacent (every pair of vertices is connected by an edge). For example, a 
$3$-clique is a triangle. Let $E_S$ be the event that a set $S$ forms a clique. 
What is the probability of $E_S$ for a particular set $S$ of $k$ vertices? 

\Part Suppose that $V_1 = \{v_1, \dots, v_{\ell}\}$ and $V_2 = \{w_1, \dots, 
w_k\}$ are two arbitrary sets of vertices. What conditions must $V_1$ and $V_2$ 
satisfy in order for $E_{V_1}$ and $E_{V_2}$ to be independent? Prove your answer.

\Part Prove that $\binom{n}{k} \le n^k$. (You might find this useful in part (e)).

\Part Prove that the probability that the graph contains a $k$-clique, for $k 
\geq 4{\log_2 n}+1$, is at most $1/n$. \textit{Hint:} Use the union bound.
\end{Parts}

\begin{solution}
    
\begin{Parts}
    
\Part The size of the sample space is $2^{\binom{n}{2}}$. The number of distinct 
pairs of vertices there are is $\binom{n}{2}$, and the two can either have an 
edge connecting the two or not. 

\Part For a set $S$ with $k$ vertices, all vertices must be connected to each
other for there to be a $k$-clique. This can only happen one way, and therefore,
$\Pr[E_S] = \frac{1}{2^{\binom{k}{2}}}$.

\Part If there are no vertices of $G$ that are both in each of the subsets $V_1$
and $V_2$, then $V_1 \cap V_2 = \emptyset$, and $\Pr[E_{V_1} \cap E_{V_2}] = 
\Pr[E_{V_1}] \cdot \Pr[E_{V_2}]$, since the edges are simply added with one half 
probability without any overlap. The probabilities are 
\[
    \begin{split}
        \Pr[E_{V_1}] &= \left( \frac{1}{2} \right)^{\binom{|V_1|}{2}} \\
        \Pr[E_{V_2}] &= \left( \frac{1}{2} \right)^{\binom{|V_2|}{2}}
    \end{split}
\]

If there are however two vertices that are both in $V_1$ and $V_2$, then there 
will be one less coin toss for both $E_{V_1}$ and $E_{V_2}$, plus the only one
coin toss for the edge that connects both of these vertices. Therefore, the 
events are also independent if there is only one shared vertex, because it 
takes two to form an edge to reduce the total number of coin tosses. 

\Part
\[
    \begin{split}
        \binom{n}{k} &= \frac{n!}{(n-k)! \cdot k!} \\
        &= \frac{(n) \times (n-1) \times \dots \times (n-k+1)}{k!} \\
        &\leq n^k
    \end{split}
\]
The numerator of the fraction multiples $n$ by itself $k$ times but is subtracted
from by many other terms and factored by $k!$ and is therefore at least less than 
or equal to $n^k$.

\Part For a graph to contain a $k$-clique, there must be a subset of vertices
of size $|S|=k$ where all vertices are connected. We know from before that the 
probability of $E_S$ is $1 / 2^{\binom{k}{2}}$. There are $\binom{n}{k}$ ways 
to pick the vertices to form $|S|$. Therefore, the probability that the graph
contains a $k$-clique is
\[
    \binom{n}{k} \times \left( \frac{1}{2} \right)^{\binom{k}{2}}
\]
Since each $S$ has that probability and there are $\binom{n}{k}$ of them. We 
know from the previous problem that $\binom{n}{k} \leq n^k$, so
\[
    \begin{split}
        \binom{n}{k} \times \left( \frac{1}{2} \right)^{\binom{k}{2}} &\leq 
        n^k \times \left( \frac{1}{2} \right)^{\binom{k}{2}} \\
        &\leq \frac{n^k}{2^{\binom{k}{2}}} \\
        &= \frac{n^k}{2^\frac{k!}{(k-2)!\cdot 2}} \\
        &= \frac{n^k}{2^\frac{k\cdot(k-1)}{2}} \\
        \text{still less than} \quad &\leq n^{(4\log_2 n + 1)} \div \left( 2^\frac{(4\log_2 n + 1) \cdot((4\log_2 n + 1)-1)}{2} \right) \\
        &= n^{(4\log_2 n + 1)} \div \left( 2^{(4\log_2 n + 1) \cdot(2\log_2 n)} \right) \\
        &= n^{(4\log_2 n + 1)} \div \left( n^{(4\log_2 n + 1) \cdot(2)} \right) \\
        &= 1 \div \left( n^{(4\log_2 n + 1)} \right) \\
        \text{again less than} \quad &\leq \frac{1}{n}
    \end{split}
\]

\end{Parts}

\end{solution}

\Question{Poisoned Smarties}

\notelinks{\href{https://www.eecs70.org/assets/pdf/notes/n14.pdf}{Note 14}}
Supposed there are 3 people who are all owners of their own Smarties factories. 
Burr Kelly, being the brightest and most innovative of the owners, produces 
considerably more Smarties than her competitors and has a commanding 50\% of the 
market share. Yousef See, who inherited her riches, lags behind Burr and produces 
40\% of the world's Smarties. Finally Stan Furd, brings up the rear with a measly 
10\%. However, a recent string of Smarties related food poisoning has forced the 
FDA investigate these factories to find the root of the problem. Through her 
investigations, the inspector found that 2 Smarties out of every 100 at Kelly's 
factory was poisonous. At See's factory, 5\% of Smarties produced were poisonous. 
And at Furd's factory, the probability a Smarty was poisonous was 0.1. 

\begin{Parts}
\Part
What is the probability that a randomly selected Smarty will be safe to eat?

\Part
If we know that a certain Smarty didn't come from Burr Kelly's factory, what is 
the probability that this Smarty is poisonous?

\Part
If a randomly selected Smarty is poisonous, what is the probability it came from 
Stan Furd's Smarties Factory?

\end{Parts}

\begin{solution}
    
\begin{Parts}
    
\Part The probability that a randomly selected Smarty is safe to eat will be 
$0.5 \times 0.98 + 0.4 \times 0.95 + 0.1 \times 0.90 =  0.96$. 

\Part The probability that a certain Smarty is poisonous, given that it is not
from Burr Kelly's factory, is
\[
    \begin{split}
        \Pr[P \mid \overline{B}] &= \frac{\Pr[P \cap \overline{B}]}{\Pr[\overline{B}]} \\
        &= \frac{\Pr[P \cap \overline{B}]}{1 - \Pr[Y] - \Pr[S]} \\
        &= \frac{0.4 \times 0.05 + 0.1 \times 0.1}{0.5} \\
        &= 0.06
    \end{split}
\]

\Part The probability that a Smary came from Stan Furd, given that it is poisonous
is, 
\[
    \begin{split}
        \Pr[S \mid P] &= \frac{\Pr[S \cap P]}{\Pr[P]} \\
        &= \frac{\Pr[P \mid S] \times \Pr[S]}{\Pr[P]} \\
        &= \frac{0.1 \times 0.1}{1 - 0.96} \\
        &= 0.25
    \end{split}
\]

\end{Parts}

\end{solution}

\Question{Symmetric Marbles}

\notelinks{\href{https://www.eecs70.org/assets/pdf/notes/n14.pdf}{Note 14}}
A bag contains 4 red marbles and 4 blue marbles. Rachel and Brooke play a game 
where they draw four marbles in total, one by one, uniformly at random, without 
replacement. Rachel wins if there are more red than blue marbles, and Brooke 
wins if there are more blue than red marbles. If there are an equal number of 
marbles, the game is tied.

\begin{Parts}
    \Part Let $A_1$ be the event that the first marble is red and let $A_2$ be 
    the event that the second marble is red. Are $A_1$ and $A_2$ independent?
    
    \Part What is the probability that Rachel wins the game?
    
    \Part Given that Rachel wins the game, what is the probability that all of 
    the marbles were red?
\end{Parts}

Now, suppose the bag contains 8 red marbles and 4 blue marbles and we add a 
tiebreaker to the game: if there are an equal number of red and blue marbles 
among the four drawn, Rachel wins if the third marble is red, and Brooke wins 
if the third marble is blue.

\begin{Parts}[resume]
    \Part What is the probability that the third marble is red?
    
    \Part Given that there are $k$ red marbles among the four drawn, where 
    $0 \leq k \leq 4$, what is the probability that the third marble is red? 
    Answer in terms of $k$.
    
    \Part Given that the third marble is red, what is the probability that 
    Rachel wins the game?
    
\end{Parts}

\pagebreak

\begin{solution}
    
\begin{Parts}
    
\Part Since we are drawing without replacement, the outcome of the first draw 
should influence the outcome of the second. If the first draw is a red marble,
the probability of $A_2$ should be lower compared to if we first drew a blue
marble. 

\Part Rachel wins the game when there are more red marbles than blue marbles,
i.e., if all the marbles are red, or if all the marbles except one are red, 
otherwise the game ends in a tie or in a loss for Rachel. There is only one 
way for all marbles to be red, which happens with probability $1 \div \binom{8}{4}$.
There are four ways for three of the marbles to be red and four different blue marbles
to choose from, therefore, the total probability of Rachel winning is 
$$\left( \binom{4}{3} \times \binom{4}{1} + 1 \right)  \div \binom{8}{4} = 
\frac{4 \times 4 + 1}{70} = \frac{17}{70}$$.

\Part The probability that all marbles were red, given that Rachel won is
\[
    \Pr[\text{all red} \mid \text{Rachel won}] = \frac{\Pr[\text{all red} \cap \text{Rachel won}]}
    {\Pr[\text{Rachel won}]} = \frac{1 \div \binom{8}{4}}{17 \div \binom{8}{4}} = \frac{1}{17}
\]

\Part The probability that third marble is red can be split into three cases: 
the first, where the first two marbles are red as well, the second where one 
of the first two is red, and the other is blue, and the third, where both of 
the first two marbles are blue. 
\[
    \begin{split}
        &\text{RR}: \quad \frac{8}{12} \times \frac{7}{11} \times \frac{6}{10} = \frac{336}{1320} \\
        &\text{RB}: \quad 2 \times \frac{8}{12} \times \frac{4}{11} \times \frac{7}{10} = \frac{448}{1320} \\
        &\text{BB}: \quad \frac{4}{12} \times \frac{3}{11} \times \frac{8}{10} = \frac{96}{1320}
    \end{split}
\]
The total probability is then $\frac{336+448+96}{1320} = \frac{880}{1320} = \frac{2}{3}$.

\Part From the previous problem's answer we can see that there is some symmetry
involved, in that, the probability that the third marble is red is the same as 
the probability that the first marble is red ($8 \div 12$). Just consider the 
first marble drawn as the third position. Since there are $k$ red marbles drawn,
with each one having the same probability of being the third as the being the first
the probability is then $\frac{k}{4}$. 

\Part For Rachel to lose the game if the third marble is red is for all other
marbles to be blue. Then the probability of her winning is one minus the probability
of her losing (considering the tie breaker rule). The probability of this is $\frac{4}{12} 
\times \frac{3}{11} \times \frac{8}{10} \times \frac{2}{9} = \frac{192}{11880}$. 
The probability that the third marble is red is $\frac{2}{3}$ from before. 
Therefore, the conditional probability is then
\[
    \Pr[\text{Rachel loses} \mid \text{Third marble is red}] = \frac{\frac{192}{11880}}
    {\frac{2}{3}} = \frac{4}{165}
\]
Then, $\Pr[\text{Rachel wins} \mid \text{Third marble is red}] = 1 - \frac{4}{165} = 
\frac{161}{165}$.

\end{Parts}

\end{solution}

\Question{Socks}
\notelinks{\href{https://www.eecs70.org/assets/pdf/notes/n13.pdf}{Note 13},
\href{https://www.eecs70.org/assets/pdf/notes/n14.pdf}{Note 14}}
Suppose you have $n$ different pairs of socks ($n$ left socks and $n$ right 
socks, for $2n$ individual socks total) in your dresser. You take the socks out 
of the dresser one by one without looking and lay them out in a row on the floor. 
In this question, we'll go through the computation of the probability that no 
two matching socks are next to each other.

\begin{Parts}
    \Part We can consider the sample space as the set of length $2n$ permutations. 
    What is the size of the sample space $\Omega$, and what is the probability 
    of a particular permutation $\omega \in \Omega$?
    \Part Let $A_i$ be the event that the $i$th pair of matching socks are next 
    to each other. Calculate $\Pr[A_i]$.
    \Part Calculate $\Pr[A_1 \cap  ... \cap A_k]$ for an arbitrary $k \geq 2$. 
    (Hint: try using a counting based approach.)
    \Part Putting these all together, calculate the probability that there is 
    at least one pair of matching socks next to each other. Your answer can (and 
    should) be expressed as a summation. (Hint: use Inclusion/Exclusion.)
    \Part Using your answer from the previous part, what is the probability that 
    no two matching socks are next to each other? (This should follow directly 
    from your answer to the previous part, and also can be left as a summation.)
\end{Parts}

\begin{solution}
    
\begin{Parts}
    
\Part The size of the sample space is $|\Omega| = (2n)!$, and any particular 
permutation has a probability of $\Pr[\omega] = \frac{1}{(2n)!}$.

\Part If two of the $2n$ socks are next to each other as a matching pair, that 
means there are only $(2n-1)!$ ways to rearrange the socks (because the pair has
to stick together). Also, the problem doesn't specify that the left one has to 
come first necessarily, so there are 2 different ways to have them stay together,
i.e., there are $2 \times (2n-1)!$ ways to order the socks with a pair staying 
together. Therefore, the $\Pr[A_i] = \frac{2 \times (2n-1)!}{(2n)!}$.

\Part For an arbitrary $k$, the probability $\Pr[A_i \cap \dots \cap A_k]$ is
\[
    \frac{2^k \times (2n-k)!}{(2n)!}
\]
since the pairs of socks have to stick together, consolidating them as one item
each to be ordered among the remaining socks. Additionally, the pairs can have 
the left and right socks be in any order presumably, and so there are $2^k$ ways
to order again. 

\Part The probability that there is at least one pair of matching socks next to
each other in the permuation is the probability that any of $A_1, A_2, \dots A_n$
happen. As such, all the permutations in each of those sets is valid, so the 
probability is then 
\[
    \begin{split}
        \Pr[A_1 \cup A_2 \cup \dots \cup A_n] &= \sum_{k=1}^{n} (-1)^{k-1} 
        \sum_{S \subseteq \{ 1, \dots, n \}: |S| = k} \Pr[\cup_{i\in S} A_i] \\
        \text{from the note} \quad &= \sum_{k=1}^{n} \Pr[A_k] - \sum_{i, j \in \{1, \dots, n\}: i < j} \Pr[A_i \cap A_j] +
        \sum_{i, j, k \in \{1, \dots, n\}: i < j < k} \Pr[A_i \cap A_j \cap A_k] - \dots \\
        &= \sum_{k=1}^{n} \frac{2 \times (2n-1)!}{(2n)!} - \sum_{i, j \in \{1, \dots, n\}: i < j} \frac{2^2 \times (2n-2)!}{(2n)!} + \dots \\
        \text{(highlight)}\quad &= \binom{n}{1} \frac{2 \times (2n-1)!}{(2n)!} - \binom{n}{2} \frac{2^2 \times (2n-2)!}{(2n)!} + \dots \\
        &= \sum_{k=1}^{n} (-1)^{k-1} \binom{n}{k} \frac{2^k \times (2n-k)!}{(2n)!}
    \end{split}
\]
Where on the highlighted line I make the observation that the sums are summing
equal probabilities, each some binomial amount. The first summation, sums $n$ 
or $\binom{n}{1}$ of the same probability, while the second sum sums $\binom{n}{2}$,
and so on. 

\Part The last part answered the question as to what the probability that there
is at least one matching pair. This question asks for the complement, i.e., that 
there are no matching pairs. The probability of this is
\[
    1 - \sum_{k=1}^{n} (-1)^{k-1} \binom{n}{k} \frac{2^k \times (2n-k)!}{(2n)!}
\]

\end{Parts}

\end{solution}

\end{document}
