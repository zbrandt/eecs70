\documentclass[11pt]{article}
\usepackage{header}
\def\title{HW 07}

\begin{document}
\maketitle
\fontsize{12}{15}\selectfont

\begin{center}
    Due: Saturday, 3/15, 4:00 PM \\
    Grace period until Saturday, 3/15, 6:00 PM \\
\end{center}

\section*{Sundry}
Before you start writing your final homework submission, state briefly how you 
worked on it.  Who else did you work with?  List names and email addresses. 
(In case of homework party, you can just describe the group.)

\begin{center}
    \textcolor{blue}{
        Zachary Brandt \\
        \nolinkurl{zbrandt@berkeley.edu}
    }
\end{center}

\vspace{15pt}

\Question{Counting on Graphs + Symmetry}
\notelinks*{\href{https://www.eecs70.org/assets/pdf/notes/n10.pdf}{Note 10}}
\begin{Parts}

    \Part How many ways are there to color the faces of a cube using exactly $6$
     colors, such that each face has a different color? Note: two colorings are 
     considered the same if one can be obtained from the other by rotating the 
     cube in any way.

    \Part How many ways are there to color a bracelet with $n$ beads using $n$ 
    colors, such that each bead has a different color? Note: two colorings are 
    considered the same if one of them can be obtained by rotating the other.

    \Part How many distinct undirected graphs are there with $n$ labeled vertices? 
    Assume that there can be at most one edge between any two vertices, and there 
    are no edges from a vertex to itself. The graphs do not have to be connected.

    \Part How many distinct cycles are there in a complete graph $K_n$ with $n$
    vertices? Assume that cycles cannot have duplicated edges. Two cycles are
    considered the same if they are rotations or inversions of each other (e.g.
    $(v_1,v_2,v_3,v_1)$, $(v_2,v_3,v_1,v_2)$ and $(v_1,v_3,v_2,v_1)$ all count as
    the same cycle).

\end{Parts}

\begin{solution}
    \begin{Parts}

        \Part The number of colorings, without considering rotations, is $6!$. 
        For a particular coloring, there are 24 other colorings that can be 
        created, just by rotating the cube. 24 because you can choose any of the 
        6 sides to start rotating the cube around 4 times, i.e., $6 \times 4$. So,
        the total number of colorings is $\frac{6!}{24} = 30$.

        \Part The number of colorings is $n!$ without considering rotations of the
        string. However, since it possible to rotate the entire string over by one 
        bead at a time to generate an equivalent but different coloring, there are $n!
        \div n = (n-1)!$ different colorings.

        \Part Each edge is determined by a pair of vertices. There are 
        $\binom{n}{2}$ distinct vertex pairs that can then determine an edge. For 
        each pair, there is either an edge, or there isn't an edge. Therefore,
        there are $2^{\binom{n}{2}}$ distinct undirected graphs with $n$ labeled
        vertices. 

        \Part $\sum_{k=3}^{n} \binom{n}{k}$ finds the number of distinct cycles there 
        are in a complete graph. The sum counts the number of length $k$ cycles, 
        up to $n$, e.g., if this was a graph with 5 vertices, it would count the 10
        distinct 3 length cycles, the 5 distinct 4 length cycles, and the one 5 length
        cycle. 

    \end{Parts}
\end{solution}

\Question{Proofs of the Combinatorial Variety}
\notelinks{\href{https://www.eecs70.org/assets/pdf/notes/n10.pdf}{Note 10}}
Prove each of the following identities using a combinatorial proof.

\begin{Parts}

\Part For every positive integer $n>1,$ 
\[\sum_{k=0}^n k \cdot \binom{n}{k} = n\cdot \sum_{k=0}^{n - 1}\binom{n - 1}{k}.\]

\Part For each positive integer $m$ and each positive integer $n > m,$
\[\sum_{a + b + c = m} \binom{n}{a}\cdot\binom{n}{b}\cdot\binom{n}{c} = \binom{3n}{m}.\]
(Notation: the sum on the left is taken over all triples of nonnegative integers 
$(a,b,c)$ such that $a + b + c = m.$)

\end{Parts}

\begin{solution}
    \begin{Parts}
        
        \Part Both sides of the equation represent the number of ways to pick out
        an element from a set with at most $n$ elements. On the left hand side, 
        each set size $k$ up to $n$ is considered, with there being $k$ ways to
        pick out an element from a set of size $k$ and $\binom{n}{k}$ ways to 
        determine that set in the first pace. On the right hand side, the selected
        element is first considered out of $n$ and then the remaining elements are
        considered to form the set, which can be up to size $n-1$ now. The
        summation all the different set sizes up to $n-1$. The right hand side can
        be read then as, pick one element from $n$, then $\binom{n-1}{0}$ sets of
        size 0, OR $\binom{n-1}{1}$ sets of size 1, OR ... and so on until $k = n-1$.


        \Part If there are three sets, $A$, $B$, and $C$, each with $n$ elements, to
        choose $m$ elements from all three of these sets, i.e., choose $m$ elements
        from $3n$, it's the same as if you choose $a$ elements from set $A$, $b$ 
        elements from set $B$, and $c$ elements from set $C$, where $a + b + c = m$.
        The sum on the left hand side represents taking into consideration all the 
        different possible ways one can choose $a$, $b$, and $c$ such that their
        sum is $m$.

    \end{Parts}
\end{solution}

\Question{Strings}

\notelinks{\href{https://www.eecs70.org/assets/pdf/notes/n10.pdf}{Note 10}}
Show your work/justification for all parts of this problem.
\begin{Parts}
    \Part How many different strings of length 5 can be constructed using the 
    characters $A, B, C$?

    \Part How many different strings of length 5 can be constructed using the 
    characters $A, B, C$ that contain at least one of each character?

\end{Parts}

\begin{solution}
    \begin{Parts}
        \Part For strings of length 5, in each of the 5 positions, there is a
        choice from 3 characters: $A$, $B$, and $C$. Therefore, there are $3^5
        = 243$ different strings that can be constructed. 
        \Part From the Inclusion/Exclusion Rule, the number of different strings
        of length 5 containing at least one of $A$, $B$, and $C$ is equal to the
        total number of strings that can be constructed, minus those with only 
        two characters, plus those with only one character, because we'll subtract
        each of AAAAA, BBBBBB, and CCCCC double in the previous step. There
        are $3 \times 2^5 = 96$ two character strings ($\times 3$ for either $(A, B)$,
        $(B, C)$, and $(A, C)$), and there are 3 ways to construct only one character 
        strings. Therefore, $243-96+3=150$ is the number of different strings that
        can be constructed from $A$, $B$, and $C$ that contain at least one of each
        character. 
    \end{Parts}
\end{solution}

\Question{Unions and Intersections}

\notelinks{\href{https://www.eecs70.org/assets/pdf/notes/n11.pdf}{Note 11}}
Given:
\begin{itemize}
\item $X$ is a countable, non-empty set. For all $i \in X$, $A_i$ is an uncountable set.
\item $Y$ is an uncountable set. For all $i \in Y$, $B_i$ is a countable set.
\end{itemize}

For each of the following, decide if the expression is
"Always countable", "Always uncountable", "Sometimes countable,
Sometimes uncountable".

For the "Always" cases, prove your claim. For the "Sometimes" cases, provide
two examples -- one where the expression is countable, and one where
the expression is uncountable.

\begin{Parts}

\Part $X \cap Y$

\begin{solution}
    This expression is always countable. There are no elements shared between 
    sets $X$ and $Y$, and the intersection will be the empty set $\emptyset$,
    which is countable. 
\end{solution}

\Part $X \cup Y$

\begin{solution}
    This expression is always uncountable. The union between these two sets will
    contain an uncountable number of sets from $Y$.
\end{solution}
	
\Part $\bigcup_{i \in X} A_i$

\begin{solution}
    The union of a countable number of uncountable sets is always uncountable
    since the expression will necessarily either be the same as any one element
    in $X$, i.e, all $A_i$ represent the same uncountable set, or a `larger' 
    uncountable set.
\end{solution}

\Part $\bigcap_{i \in X} A_i$

\begin{solution}
    The intersection of a countable number of uncountable sets is sometimes
    countable. Consider the case where each $A_i$ represents a different 
    subset of the real numbers, e.g., $A_1$ numbers between 0 and 1, $A_2$
    numbers between 1 and 2, etc. The intersection between all these sets 
    will be the empty set $\emptyset$ which is countable. It could then also
    be the case that the intersection is uncountable, if all the countable
    number of sets $A_i$ were the same uncountable set. 
\end{solution}

\Part $\bigcup_{i \in Y} B_i$

\begin{solution}
    The union of an uncountable number of countable sets is sometimes countable.
    For example, if each set $A_i$ is the same countable set, e.g., the natural
    numbers, then the union will be this same countable set. However, it's
    possible for the union to produce an uncountable set as well. For example, 
    if there are an uncountable number of sets, each set $A_i$ could just contain
    one real number from the subset of the reals between 0 and 1. The union would
    then be the numbers in that range.
\end{solution}

\Part $\bigcap_{i \in Y} B_i$

\begin{solution}
    Using the last example from above, it is possible to construct a countable 
    empty set via the intersection of an uncountable number of $A_i$, where 
    each $A_i$ simply contains a different number from the subset of the real 
    numbers between 0 and 1. This is always countable, however, since any 
    intersection of countable sets will yield another countable set.
\end{solution}


\end{Parts}

\Question{Count It!}

\notelinks{\href{https://www.eecs70.org/assets/pdf/notes/n11.pdf}{Note 11}}
For each of the following collections, determine and briefly explain whether it
is finite, countably infinite (like the natural numbers), or uncountably infinite 
(like the reals):

\begin{Parts}

\Part The integers which divide $8$.

\begin{solution}
    This collection is finite since all numbers that divide 8 must be less than 
    or equal to 8 and greater than or equal to 1. 
\end{solution}

\Part The integers which $8$ divides.

\begin{solution}
    This collection is countably infinite as it is possible to enumerate the 
    integers which 8 divides, i.e., there is a bijection between this set and
    the natural numbers. The mapping from the natural numbers to this set is 
    defined by $f(x) = 8x$, where $x \in \mathbb{N}$.
\end{solution}

\Part The functions from $\mathbb{N}$ to $\mathbb{N}$.

\begin{solution}
    This collection is uncountably infinite, like the reals. Assuming that 
    it is possible to enumerate all functions $f: \mathbb{N} \to \mathbb{N}$,
    it is then possible to construct a new function not on this list by adding 
    $+1$ to each function. Therefore, it is not possible to enumerate, and is 
    uncountably infinite. 
\end{solution}

\Part The set of strings over the English alphabet. (Note that the strings may 
be arbitrarily long, but each string has finite length. Also the strings need 
not be real English words.)

\begin{solution}
    Since each string is of finite length, like a natural number, and since it 
    is possible to express each string as a natural number ($A$ as 1, $B$ as 2, 
    etc.), the set of strings over the English alphabet is countably infinite. 
\end{solution}

\Part The set of finite-length strings drawn from a countably infinite alphabet, 
$\mathcal{C}$.

\begin{solution}
    With a countably infinite alphabet, it is possible to enumerate each letter 
    in the alphabet, i.e., pair it with a corresponding integer in the natural 
    numbers. Each string in the set of finite-length strings over this alphabet
    can then be expressed as a finite natural number and the collection is therefore
    countably infinite. 
\end{solution}

\Part The set of infinite-length strings over the English alphabet.

\begin{solution}
    This set is uncountably infinite, like the reals, because it can be expressed
    as a subset of the real numbers between 0 and 1. Again, every letter in each
    string can be mapped to an integer in the natural numbers. Each of these
    infinite-lengthed numbers can then be appended with a `0.' at the start. The
    crucial difference with this question is that each string is now of infinite 
    length, making it uncountably infinite, through its parralel with the real 
    numbers. 
\end{solution}

\end{Parts}

\end{document}
