\begin{homeworkProblem}{Grid Induction}

    Pacman is walking on an infinite 2D grid. He starts at some location $(i, j) \in \mathbb{N}^2$ in the first quadrant, and is constrained to stay in the first quadrant (say, by walls along the $x$ and $y$ axes). 
    \\ \\
    Every second he does one of the following (if possible):
    \begin{enumerate}
        \item[(i)] Walk one step down, to $(i, j-1)$
        \item[(ii)] Walk one step left, to $(i-1, j)$
    \end{enumerate}
    
    For example, if he is at (5, 0), his only option is to walk left to (4, 0); if Pacman is instead at (3, 2), he could walk either to (2, 2) or (3, 1).
    \\ \\
    Prove by induction that no matter how he walks, he will always reach (0, 0) in finite time
    \\ \\
    \textbf{Justification}

    To strengthen the hypothesis, I will say that no matter how Pacman walks, he will reach (0, 0) in finite time, specifically $i+j$ seconds. 
    \\ \\
    For the base case, assume he starts at (0, 0), then he already is at the end in zero seconds, which the math also shows, $i+j=0+0=0$ as well.
    \\ \\
    For the inductive hypothesis, assume that this theory is true for all locations in $\mathbb{N}^2$ in the first quadrant less than or equal to some arbitrary $(i, j)$. That is, for locations where the x-position is less than or equal to $i$ and where the y-position is less than or equal to $j$.
    \\ \\
    For the inductive step, I'll prove that a location one step removed, either $(i+1, j)$ or $(i, j+1)$, from this aformentioned location also reaches (0, 0) in finite time, specifically $i+j+1$ seconds. We have two cases, either he walks down one or walks one to the left. If he is at $(i+1, j)$ and moves one to the left, then we are back at $(i, j)$ which we assume reaches (0, 0) in $i+j$ time, for a total of $i+j+1$ seconds. If he moves one down to $(i +1, j-1)$, he will then either have to move to the left in the next step, in which case he will be in a position where we know he'll reach (0, 0) in a total finite time of $i+j+1$ seconds, or move down in one second. Eventually, he will either have to go left or move down in $j$ seconds (once he reaches the x-axis) and go left in $i$ seconds for a total time of $i+j+1$ seconds. This whole situation applies vice versa for if he started at $(i, j+1)$ instead. 
    \\ \\
    Since we have shown with strong induction and a strengthened hypothesis that for any position $(i, j) \in \mathbb{N}^2$ in the first quadrant he will reach (0, 0) in $i+j$ seconds, he will always reach (0, 0) in finite time. 
    
    
\end{homeworkProblem}