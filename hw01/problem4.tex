\begin{homeworkProblem}{Twin Primes}
    \begin{enumerate}
        \item[A)] Let $p > 3$ be a prime. Prove that $p$ is of the form $3k+1$ or $3k-1$ for some integer $k$.
        \item[B)] \textit{Twin primes} are pairs of prime numbers $p$ and $q$ that have a difference of 2. Use part A) to prove that 5 is the only prime number that takes part in two different twin prime pairs. 
    \end{enumerate}

    \textbf{Part A} \\
    Every integer can be expressed as either one of $3k$, $3k+1$, or $3k+2$ (going on higher gets back to $3k$ but with $k+1$) for some integer $k$. Prime numbers are those integers that can only be factored by themselves and one. Therefore, $3k$ cannot express a prime number. However, $3k+1$ and $3k+2=3(k+1)-1=3l-1$ can express prime numbers. 
    \\ \\
    \textbf{Part B} \\
    To be part of two twin primes, that means there is a number $p$, which is prime, and two numbers $p-2$ and $p+2$ that are also prime. Since they are all seperated by 2, that means they must all be odd, since $p$ cannot be even (otherwise it would be divisible by 2). Since $p$ is either of the form $3k+1$ or $3k-1$, either $p-2$ or $p+2$ is divisible by 3. In case 1, where $p = 3k+1$ for some integer $k$, the supposed prime $p+2$ has to be divisible by 3, $p+2=3k+1+2= 3k+3$. In case 2, where $p=3k-1$ for some integer $k$, the supposed prime $p-2$ has to be divisible by 3, $p-2=3k-1-2=3k-3$. Therefore, for all of $p-2$, $p$, and $p+2$ to be prime, one of them has to be the number 3. Therefore, $p-2$ has to be the number 3. Therefore, $p$ has to be 5. 
    
\end{homeworkProblem}