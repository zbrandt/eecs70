\begin{homeworkProblem}{Fermat's Wristband}

Let $p$ be a prime number and let $n$ be a positive integer. We have 
beads of $n$ different colors, where any two beads of the same color 
are indistinguishable.

\begin{itemize}
    \item[A)]
    We place $p$ beads onto a string. How many different ways are there 
    to construct such a sequence of $p$ beads with up to $n$ different 
    colors?

    \item[B)] 
    How many sequences of $p$ beads on the string are there that use at 
    least two colors?

    \item[C)]
    Now we tie the two ends of the string together, forming a wristband. 
    Two wristbands are equivalent if the sequence of colors on one can 
    be obtained by rotating the beads on the other. (For instance, if we 
    have $n=3$ colors, red (R), green (G), and blue (B), then the length 
    $p = 5$ necklaces RGGBG, GGBGR, GBGRG, BGRGG, and GRGGB are all 
    equivalent, because these are all rotated versions of each other.)

    How many non-equivalent wristbands are there now? Again, the $p$ 
    beads must not all have the same color. (Your answer should be a 
    simple function of $n$ and $p$.)

    [\textit{Hint}: Think about the fact that rotating all the beads on 
    the wristband to another position produces an identical wristband.]

    \item[D)] Use your answer to part (c) to prove Fermat's little 
    theorem.
\end{itemize}

\end{homeworkProblem}
