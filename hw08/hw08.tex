\documentclass[11pt]{article}
\usepackage{header}
\def\title{HW 08}

\begin{document}
\maketitle
\fontsize{12}{15}\selectfont

\begin{center}
    Due: Saturday, 3/22, 4:00 PM \\
    Grace period until Saturday, 3/22, 6:00 PM \\
\end{center}

\section*{Sundry}
Before you start writing your final homework submission, state briefly how you worked on it.  Who else did you work with?  List names and email addresses.  (In case of homework party, you can just describe the group.)

\vspace{15pt}

\Question{Unprogrammable Programs}

\notelinks{\href{https://www.eecs70.org/assets/pdf/notes/n12.pdf}{Note 12}}
Prove whether the programs described below can exist or not.

\begin{Parts}

\Part A program $P(F,x,y)$ that returns true if the program $F$ outputs $y$ when given $x$ as input (i.e. $F(x)=y$) and false otherwise.

\Part A program $P$ that takes two programs $F$ and $G$ as arguments, and returns true if for all inputs $x$, $F$ halts on $x$ iff $G$ halts on $x$ (and returns false if this equivalence is not always true).

\textit{Hint:} Use $P$ to solve the halting problem, and consider defining two subroutines to pass in to $P$, where one of the subroutines always loops.

\end{Parts}

\Question{Kolmogorov Complexity}

\notelinks{\href{https://www.eecs70.org/assets/pdf/notes/n12.pdf}{Note 12}}
Compressing a bit string $x$ of length $n$ can be interpreted as the task of creating a program of fewer than $n$ bits that returns $x$.
The Kolmogorov complexity of a string $K(x)$ is the length of an optimally-compressed copy of $x$; that is, $K(x)$ is the length of shortest program that returns $x$.

\begin{Parts}

\Part
Explain why the notion of the "smallest positive integer that cannot be defined in under 280 characters" is paradoxical.

\Part
Prove that for any length $n$, there is at least one string of bits that cannot be compressed to less than $n$ bits, assuming that no two strings can be compressed to the same value.

\Part
Say you have a program $K$ that outputs the Kolmogorov complexity of any input
string. Under the assumption that you can use such a program $K$ as a
subroutine, design another program $P$ that takes an integer $n$ as input, and
outputs the length-$n$ binary string with the highest Kolmogorov complexity. If
there is more than one string with the highest complexity, output the one that
comes first lexicographically.

\Part
Let's say you compile the program $P$ you just wrote and get an $m$ bit
executable, for some $m \in \mathbb N$ (i.e. the program $P$ can be represented
in $m$ bits). Prove that the program $P$ (and consequently the program $K$)
cannot exist.

(\textit{Hint}: Consider what happens when $P$ is given a very large input $n$ that is much greater than $m$.)

 

\end{Parts}

\Question{Five Up}

\notelinks{\href{https://www.eecs70.org/assets/pdf/notes/n13.pdf}{Note 13}}
Say you toss a coin five times, and record the outcomes. For the three questions
below, you can assume that order matters in the outcome, and that the
probability of heads is some $p$ in $0 < p < 1$, but \textit{not} that the coin
is fair ($p = 0.5$).

\begin{Parts}
\Part What is the size of the sample space, $|\Omega|$?

\Part How many elements of $\Omega$ have exactly three heads?

\Part How many elements of $\Omega$ have three or more heads?
\end{Parts}

For the next three questions, you can assume that the coin is fair (i.e. heads comes up with $p=0.5$, and tails otherwise).

\begin{Parts}[resume]
\Part What is the probability that you will observe the sequence HHHTT? What about HHHHT?

\Part What is the probability of observing at least one head?

\Part What is the probability you will observe more heads than tails?

\end{Parts}

\begin{solution}

\begin{Parts}
\Part The size of the sample space $\Omega$ is $2^5=32$. Elements in the sample
space are of the type $\omega \in \mathbb{R}^5$.

\Part For three heads out of five coin flips, this subset of $\Omega$ has size
$\binom{5}{3}=10$.

\Part This subset is the union of the subsets of eleemnts with only 3, 4, and 5
heads each, i.e., $\binom{5}{3} + \binom{5}{4} + \binom{5}{5} = 10 + 5 + 1 = 16$.

\Part For both sequences, the probability of observing each is $\frac{1}{2^5} =
\frac{1}{32}$.

\Part The complement of the subset of elements with at least one head is the 
subset of elements containing no heads. Therefore, the probability of observing
a sequence with at least one head is $1 - \frac{1}{32} = \frac{31}{32}$, since
there is only one sequence without any heads (TTTTT).

\Part For there to be more heads than tails, there must be at most two tails
in the sequence. The number of elements with this property is $\binom{5}{0}
+ \binom{5}{1} + \binom{5}{2} = 10 + 5 + 1 = 16$. The probability of 
observing such a sequence is then $\frac{16}{32} = \frac{1}{2}$.


\end{Parts}

\end{solution}

\Question{Aces}

\notelinks{\href{https://www.eecs70.org/assets/pdf/notes/n13.pdf}{Note 13}}
Consider a standard 52-card deck of cards, which has 4 suits (hearts, diamonds, clubs, and spades) with 13 cards in each suit. Each suit has one ace. Hearts and diamonds are red, while clubs and spades are black. 
\begin{Parts}
    \Part Find the probability of getting an ace or a red card, when drawing a 
    single card.
    
    \Part Find the probability of getting an ace or a spade, but not both, when 
    drawing a single card.
    
    \Part Find the probability of getting the ace of diamonds when drawing a 5 
    card hand.
    
    \Part Find the probability of getting exactly 2 aces when drawing a 5 card 
    hand.
    
    \Part Find the probability of getting at least 1 ace when drawing a 5 card 
    hand.
    
    \Part Find the probability of getting at least 1 ace or at least 1 heart when 
    drawing a 5 card hand.
    
\end{Parts}

\Question{Past Probabilified}
\notelinks{\href{https://www.eecs70.org/assets/pdf/notes/n13.pdf}{Note 13}}
In this question we review some of the past CS70 topics, and look at them probabilistically.\\
For the following experiments, define an appropriate sample space $\Omega$, and give the probability function $\mathbb P[\omega]$ for each $\omega \in \Omega$. Then compute the probabilities of the events $E_1$ and $E_2$.\\

\begin{Parts}
    \Part Fix a prime $p>2$, and uniformly sample twice with replacement from
    $\{0, \dots, p-1\}$ (assume we have two $\{0, \dots, p-1\}$-sided fair dice and we roll them). Then multiply these two numbers with each other in
    $(\bmod{p})$ space.\\
    \begin{center}
    $E_1 =$ The resulting product is $0$.\\
    $E_2 =$ The product is $(p-1)/2$.
    \end{center}

    \Part Make a graph on $n$ vertices by sampling uniformly at random from all possible edges, (assume for each edge we flip a coin and if it is head we include the edge in the graph and otherwise we exclude that edge from the graph).\\
    \begin{center}
    $E_1 =$ The graph is complete.\\
    $E_2 =$ vertex $v_1$ has degree $d$.
    \end{center}

    \Part Create a random stable matching instance by having each person's
    preference list be a random permutation of the opposite entity's list (make the preference list for each individual job and each individual candidate a random permutation of the opposite entity's list). Finally, create a uniformly random pairing by matching jobs and candidates up uniformly at random (note that in this pairing, (1) a candidate cannot be matched with two different jobs, and a job cannot be matched with two different candidates (2) the pairing does not have to be stable).\\
    \begin{center}
     $E_1 =$ All jobs have distinct favorite candidates.\\
     $E_2 =$ The resulting pairing is the candidate optimal stable pairing.
    \end{center}

\end{Parts}

\end{document}
