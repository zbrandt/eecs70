\begin{homeworkProblem}{Lagrange? More like Lamegrange.}

    \begin{itemize}

        \item[A)] Let's say we wanted to interpolate a polynomial through a 
        single point, $(x_0, y_0)$. What would be the polynomial that we would 
        get? (This is not a trick question. A degree 0 polynomial is fine.)
        
        \item[B)] Call the polynomial from the previous part $f_0(x)$. Now say 
        we wanted to define the polynomial $f_1(x)$ that passes through the 
        points $(x_0, y_0)$ and $(x_1, y_1)$. If we write $f_1(x) = f_0(x) + 
        a_1(x - x_0)$, what value of $a_1$ causes $f_1(x)$ to pass through the 
        desired points?
        
        \item[C)] Now say we want a polynomial $f_2(x)$ that passes through 
        $(x_0, y_0)$, $(x_1, y_1)$, and $(x_2, y_2)$. If we write $f_2(x) = 
        f_1(x) + a_2(x - x_0)(x - x_1)$, what value of $a_2$ gives us the 
        desired polynomial?
        
        \item[D)] Suppose we have a polynomial $f_i(x)$ that passes through the 
        points $(x_0, y_0)$, ..., $(x_i, y_i)$ and we want to find a polynomial 
        $f_{i + 1}(x)$ that passes through all those points and also $(x_{i + 
        1}, y_{i + 1})$. If we define $f_{i + 1}(x) = f_i(x) + a_{i + 1}\prod_{j 
        = 0}^i (x - x_j)$, what value must $a_{i + 1}$ take on?

    \end{itemize}

    \part 

    If we interpolate a polynomial through a single point $(x_0, y_0)$, we get
    a 0 degree polynomial, i.e., the point itself for all $x$: $f_0(x) = y_0$.

    \part 

    When $x=x_0$, $f_1(x)$ is equal to $f_1(x_0)= f_0(x_0) + a_1 (x_0 - x_0)
    = y_0$. Then for $x=x_1$, $f_1(x_1) = f_0(x_1) + a_1 (x_1 - x_0) = y_0 +
    a_1 (x_1 - x_0)$. $f_1(x_1)$ should equal $y_1$, so 
    \[
        \begin{split}
            y_1 &= y_0 + a_1 (x_1 - x_0) \\
            a_1 &= \frac{y_1 - y_0}{x_1 - x_0} = \frac{y_1 - f_0(x_1)}{x_1 - x_0}
        \end{split}
    \]

    \part 

    For $f_2(x)$ evaluated at $x=x_0$ and $x=x_1$, we know that $f_1(x)$ will
    output our desired values of $y_0$ and $y_1$, and since the rest of the 
    terms in $f_2(x)$ evaluate to zero, we know $f_2(x)$ passes through the 
    first two of our points. For $x=x_2$, $f_2(x)$ is $f_2(x_2) = f_1(x_2)
    + a_2 (x_2 -x_0)(x_2-x_1)$. Therefore, $a_2$ must be
    \[
        a_2 = \frac{y_2-f_1(x_2)}{(x_2 - x_0)(x_2-x_1)}
    \]

    \part 

    Like in the previous parts, for values $x_0, x_1, \dots, x_i$, $f_{i+1}(x)$
    will output $y_0, y_1, \dots, y_i$ since the product $\prod_{j=0}^{i}(x-x_j)$
    will end up being zero and $f_{i+1}(x)$ collapses to $f_i(x)$. Like before, 
    consider $f_{i+1}(x_{i+1}) = f_i(x_{i+1}) + a_{i+1} \prod_{j=0}^{i}(x_{i+1}
    - x_j)$ and that it should equal $y_{i+1}$. Therefore, $a_{i+1}$ is
    \[
        \begin{split}
            y_{i+1} &= f_i(x_{i+1}) + a_{i+1} \prod_{j=0}^{i} (x_{i+1} - x_j) \\
            a_{i+1} &= \frac{y_{i+1} - f_{i}(x_{i+1})}{\prod_{j=0}^{i}(x_{i+1}-x_j)}
        \end{split}
    \]


\end{homeworkProblem}
