\begin{homeworkProblem}{Lagrange? More like Lamegrange.}

    \begin{itemize}

        \item[A)] Let's say we wanted to interpolate a polynomial through a 
        single point, $(x_0, y_0)$. What would be the polynomial that we would 
        get? (This is not a trick question. A degree 0 polynomial is fine.)
        
        \item[B)] Call the polynomial from the previous part $f_0(x)$. Now say 
        we wanted to define the polynomial $f_1(x)$ that passes through the 
        points $(x_0, y_0)$ and $(x_1, y_1)$. If we write $f_1(x) = f_0(x) + 
        a_1(x - x_0)$, what value of $a_1$ causes $f_1(x)$ to pass through the 
        desired points?
        
        \item[C)] Now say we want a polynomial $f_2(x)$ that passes through 
        $(x_0, y_0)$, $(x_1, y_1)$, and $(x_2, y_2)$. If we write $f_2(x) = 
        f_1(x) + a_2(x - x_0)(x - x_1)$, what value of $a_2$ gives us the 
        desired polynomial?
        
        \item[D)] Suppose we have a polynomial $f_i(x)$ that passes through the 
        points $(x_0, y_0)$, ..., $(x_i, y_i)$ and we want to find a polynomial 
        $f_{i + 1}(x)$ that passes through all those points and also $(x_{i + 
        1}, y_{i + 1})$. If we define $f_{i + 1}(x) = f_i(x) + a_{i + 1}\prod_{j 
        = 0}^i (x - x_j)$, what value must $a_{i + 1}$ take on?

    \end{itemize}

\end{homeworkProblem}
