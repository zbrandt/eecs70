\documentclass[11pt]{article}
\usepackage{header}
\def\title{HW 05}

\begin{document}
\maketitle
\fontsize{12}{15}\selectfont

\begin{center}
    Due: Saturday, 3/1, 4:00 PM \\
    Grace period until Saturday, 3/1, 6:00 PM \\
\end{center}

\section*{Sundry}
Before you start writing your final homework submission, state briefly how you worked on it.  Who else did you work with?  List names and email addresses.  (In case of homework party, you can just describe the group.)

\vspace{15pt}

\Question{Equivalent Polynomials}

\notelinks{\href{https://www.eecs70.org/assets/pdf/notes/n7.pdf}{Note 7},\href{https://www.eecs70.org/assets/pdf/notes/n8.pdf}{Note 8}}
This problem is about polynomials with coefficients in $\text{GF}(p)$ for some prime $p \in \mathbb{N}$. We say that two such polynomials $f$ and $g$ are \emph{equivalent} if $f(x) \equiv g(x) \pmod{p}$ for every $x \in \text{GF}(p)$. 

\begin{Parts}
    \Part Show that $f(x)=x^{p-1}$ and $g(x)=1$ are \textbf{not} equivalent polynomials under $\text{GF}(p).$

    \Part Use Fermat's Little Theorem to find a polynomial with degree strictly less than 5 that is equivalent to $f(x) = x^5$ over $\text{GF}(5)$; then find a polynomial with degree strictly less than 11 that is equivalent to $g(x) = 4x^{70} + 9x^{11} + 3$ over $\text{GF}(11)$.

    \Part In $\mathrm{GF}(p)$, prove that whenever $f(x)$ has degree $\ge p$, it is equivalent to some polynomial $\tilde f(x)$ with degree $< p$.
\end{Parts}

\Question{Secret Sharing}
\notelinks{\href{https://www.eecs70.org/assets/pdf/notes/n8.pdf}{Note 8}}
Suppose the Oral Exam questions are created by 2 TAs and 3 Readers. The answers are all encrypted, and we know that:

\begin{itemize}
\item Two TAs together should be able to access the answers
\item Three Readers together should be able to access the answers
\item One TA and one Reader together should also be able to access the answers
\item One TA by themself or two Readers by themselves should not be able to access the answers.
\end{itemize}

Design a Secret Sharing scheme to make this work.

\Question{To The Moon!}

\notelinks{\href{https://www.eecs70.org/assets/pdf/notes/n8.pdf}{Note 8}}
A secret number $s$ is required to launch a rocket, and Alice distributed the values \\ \(\left(1,p(1)\right), \left(2,p(2)\right), \dots, \left(n+1,p(n+1)\right)\) of a degree $n$  polynomial $p(x)$ to a group of \$GME holders $\text{Bob}_1, \dots, \text{Bob}_{n+1}$. As usual, she chose $p(x)$ such that $p(0) = s$. $\text{Bob}_1$ through $\text{Bob}_{n+1}$ now gather to jointly discover the secret. However, $\text{Bob}_1$ is secretly a partner at Melvin Capital and already knows $s$, and wants to sabotage $\text{Bob}_2,\dots, \text{Bob}_{n+1}$, making them believe that the secret is in fact some fixed $s' \neq s$. How could he achieve this? In other words, what value should he report (in terms variables known in the problem, such as $s', s$ or $p(1)$) in order to make the others believe that the secret is $s'$?

\Question {Lagrange?  More like Lamegrange.}  
\notelinks{\href{https://www.eecs70.org/assets/pdf/notes/n8.pdf}{Note 8}}
In this problem, we walk you through an alternative to Lagrange interpolation.

\begin{Parts}

	\Part Let's say we wanted to interpolate a polynomial through a single point, $(x_0, y_0)$.  What would be the polynomial that we would get?  (This is not a trick question. A degree 0 polynomial is fine.)
	
	\Part Call the polynomial from the previous part $f_0(x)$.  Now say we wanted to define the polynomial $f_1(x)$ that passes through the points $(x_0, y_0)$ and $(x_1, y_1)$.  If we write $f_1(x) = f_0(x) + a_1(x - x_0)$, what value of $a_1$ causes $f_1(x)$ to pass through the desired points?
	
	\Part Now say we want a polynomial $f_2(x)$ that passes through $(x_0, y_0)$, $(x_1, y_1)$, and $(x_2, y_2)$.  If we write $f_2(x) = f_1(x) + a_2(x - x_0)(x - x_1)$, what value of $a_2$ gives us the desired polynomial?
	
	\Part Suppose we have a polynomial $f_i(x)$ that passes through the points $(x_0, y_0)$, ..., $(x_i, y_i)$ and we want to find a polynomial $f_{i + 1}(x)$ that passes through all those points and also $(x_{i + 1}, y_{i + 1})$.  If we define $f_{i + 1}(x) = f_i(x) + a_{i + 1}\prod_{j = 0}^i (x - x_j)$, what value must $a_{i + 1}$ take on?
	
	
\end{Parts}

\Question{Error-Correcting Codes}

\notelinks*{\href{https://www.eecs70.org/assets/pdf/notes/n9.pdf}{Note 9}}
\begin{Parts}
  
\Part Recall from class the error-correcting code for erasure errors, which
protects against up to $k$ lost packets by sending a total of $n+k$ packets
(where $n$ is the number of packets in the original message).  Often the number
of packets lost is not some fixed number $k$, but rather a \emph{fraction} of
the number of packets sent.  Suppose we wish to protect against a fraction
$\alpha$ of lost packets (where $0 < \alpha < 1$).  At least how many packets do 
we need to send (as a function of $n$ and $\alpha$)?

\Part 
Repeat part (a) for the case of general errors.
\end{Parts}

\Question{Alice and Bob}

\notelinks*{\href{https://www.eecs70.org/assets/pdf/notes/n8.pdf}{Note 8},\href{https://www.eecs70.org/assets/pdf/notes/n9.pdf}{Note 9}}
\begin{Parts}
	\Part Alice decides that instead of encoding her message as the values of a polynomial, she will encode her message as the coefficients of a degree $2$ polynomial $P(x)$. For her message [$m_1, m_2, m_3$], she creates the polynomial $P(x) = m_1x^2 + m_2x + m_3$ and sends the five packets $(0, P(0))$, $(1, P(1))$, $(2, P(2))$, $(3, P(3))$, and $(4, P(4))$ to Bob. However, one of the packet $y$-values (one of the $P(i)$ terms; the second attribute in the pair) is changed by Eve before it reaches Bob. If Bob receives
    	\begin{align*}
      		(0, 1), (1, 3), (2, 0), (3, 1), (4, 0)
    	\end{align*}
    and knows Alice's encoding scheme and that Eve changed one of the packets, can he recover the original message? If so, find it as well as the $x$-value of the packet that Eve changed. If he can't, explain why. Work in mod 7. Also, feel free to use a calculator or online systems of equations solver, but make sure it can work under mod 7.
   
	\Part Bob gets tired of decoding degree 2 polynomials. He convinces Alice to encode her messages on a degree 1 polynomial. Alice, just to be safe, continues to send 5 points on her polynomial even though it is only degree 1. She makes sure to choose her message so that it can be encoded on a degree 1 polynomial. However, Eve changes two of the packets. Bob receives $(0, 5)$, $(1, 7)$, $(2, x)$, $(3, 5)$, $(4, 0)$. If Alice sent $(0, 5)$, $(1, 7)$, $(2, 9)$, $(3, -2)$, $(4, 0)$, for what values of $x$ will Bob not uniquely be able to determine Alice's message? Assume that Bob knows Eve changed two packets. Work in mod 13. Again, feel free to use a calculator or graphing calculator software.

\textit{Hint:} Observe that since Bob knows that Eve changed two packets, he's looking for a polynomial that passes through at least 3 of the given points. Think about what must happen in order for Bob to be unable to uniquely identify the original polynomial.

	\Part
	Alice wants to send a length $n$ message to Bob. There are two communication channels available to her: Channel X and Channel Y. Only 6 packets can be sent through channel X. Similarly, Channel Y will only deliver 6 packets, but it will also corrupt (change the value) of one of the delivered packets. Using each of the two channels once, what is the largest message length $n$ such that Bob so that he can always reconstruct the message? 

\end{Parts}

\end{document}
