\begin{homeworkProblem}{Alice and Bob}

    \begin{itemize}
        \item[A)] Alice decides that instead of encoding her message as the 
        values of a polynomial, she will encode her message as the coefficients 
        of a degree $2$ polynomial $P(x)$. For her message [$m_1, m_2, m_3$], 
        she creates the polynomial $P(x) = m_1x^2 + m_2x + m_3$ and sends the 
        five packets $(0, P(0))$, $(1, P(1))$, $(2, P(2))$, $(3, P(3))$, and 
        $(4, P(4))$ to Bob. However, one of the packet $y$-values (one of the 
        $P(i)$ terms; the second attribute in the pair) is changed by Eve before 
        it reaches Bob. If Bob receives
        \begin{align*}
            (0, 1), (1, 3), (2, 0), (3, 1), (4, 0)
        \end{align*}
        and knows Alice's encoding scheme and that Eve changed one of the 
        packets, can he recover the original message? If so, find it as well as 
        the $x$-value of the packet that Eve changed. If he can't, explain why. 
        Work in mod 7. Also, feel free to use a calculator or online systems of 
        equations solver, but make sure it can work under mod 7.
       
        \item[B)] Bob gets tired of decoding degree 2 polynomials. He convinces 
        Alice to encode her messages on a degree 1 polynomial. Alice, just to be 
        safe, continues to send 5 points on her polynomial even though it is 
        only degree 1. She makes sure to choose her message so that it can be 
        encoded on a degree 1 polynomial. However, Eve changes two of the 
        packets. Bob receives $(0, 5)$, $(1, 7)$, $(2, x)$, $(3, 5)$, $(4, 0)$. 
        If Alice sent $(0, 5)$, $(1, 7)$, $(2, 9)$, $(3, -2)$, $(4, 0)$, for 
        what values of $x$ will Bob not uniquely be able to determine Alice's 
        message? Assume that Bob knows Eve changed two packets. Work in mod 13. 
        Again, feel free to use a calculator or graphing calculator software.

        \textit{Hint:} Observe that since Bob knows that Eve changed two 
        packets, he's looking for a polynomial that passes through at least 3 of 
        the given points. Think about what must happen in order for Bob to be 
        unable to uniquely identify the original polynomial.

        \item[C)] Alice wants to send a length $n$ message to Bob. There are two 
        communication channels available to her: Channel X and Channel Y. Only 6 
        packets can be sent through channel X. Similarly, Channel Y will only 
        deliver 6 packets, but it will also corrupt (change the value) of one of 
        the delivered packets. Using each of the two channels once, what is the 
        largest message length $n$ such that Bob so that he can always 
        reconstruct the message? 

        \end{itemize}

        \part 

        To recover the original message, Bob will have to recover the polynomial $P(x)$
        using the Berlekamp–Welch Algorithm. Since we know that Eve only changed one of 
        the packets, the error locator polynomial is $E(x) = (x-e_1) = e + b_0$, as there 
        is only one position that is in error. The system of equations to solve for $b_0$
        is given by $Q(i) = P(i)E(i)$ for $0 \leq i \leq 4$ where $Q(x) = P(x)Q(x)$ is 
        of the form $Q(x) = a_3 \cdot x^3 + a_2 \cdot x^2 + a_1 \cdot x + a_0$ since 
        $P(x)$ is of degree 2 and $E(x)$ is of degree 1.
        
        \begin{align*}
            &(i=0) & a_0 + 6b_0 \equiv 0 \pmod{7} \\
            &(i=1) & a_3 + a_2 + a_1 + a_0 + 4b_0 \equiv 3 \pmod{7} \\
            &(i=2) & a_3 + 4a_2 + 2a_1 + a_0 \equiv 0 \pmod{7} \\
            &(i=3) & 6a_3 + 2a_2 + 3a_1 + a_0 + 6b_0 \equiv 3 \pmod{7} \\
            &(i=4) & a_3 + 2a_2 + 4a_1 + a_0 \equiv 0 \pmod{7}
        \end{align*}

        To solve this, we can row reduce a corresponding matrix for the system of
        equations under modulos 7. 
        \[
        \left[
            \begin{array}{ccccc|c}
                0 & 0 & 0 & 1 & 6 & 0 \\
                1 & 1 & 1 & 1 & 4 & 3 \\
                1 & 4 & 2 & 1 & 0 & 0 \\
                6 & 2 & 3 & 1 & 6 & 3 \\
                1 & 2 & 4 & 1 & 0 & 0
            \end{array}
        \right]
        \]

        Solving this (on paper), the coefficients are $a_3=1, a_2=5, a_1=5, a_0=4,$ 
        and $b_0=-3$. Therefore, the polynomial $Q(x)$ is $Q(x) = x^3 + 5x^2 + 5x
        + 4$ and the error location is $e_1 = -b_0 = 3$. The secret message
        is the coefficients of $P(x)$ which is 
        \[
            P(x) = \frac{Q(x)}{E(x)} = \frac{x^3 + 5x^2 + 5x + 4}{x - 3} = 
            x^2+8x+29 \; \textit{(R: 91)} \equiv x^2 + x + 1 \pmod{7}
        \]

        So the secret message is then $m_1 = 1, m_2 = 2, m_3 = 1$ and the 
        error is located at $i=3$.

        \part 
        
        If Alice is encoding her messags on a 1 degree polynomial, all the 
        points she attempts to send Bob must fall on the same line. For the 
        corrupted message, Bob can still find the 1 degree polynomial from the 
        3 of the 5 points that are uncorrupted and still fall on the same line.
        So that Bob can no longer \textit{uniquely} determine a 1 degree polynomial
        with the corrupted packets, there must be at least two sets of three
        points that determine two different lines. 
        \\ \\
        The points $(0, 5), (1, 7), (4, 0)$ fall on Alice's 1 degree polynomial 
        under modulo 13. An $x$ value that determines another line is $x=10$. 
        Although it falls on Alice's original line, in conjunction with $(4,0)$
        and the corrupted $(3, 5)$ it forms another line with slope $m=-5$. $(3, 5)$
        is also at the same y-level as the point $(0, 5)$, and so if $x=5$,
        an additional, horizontal line is determined with a set of three points.
        Finally, $x=6$ determines another line with points $(1, 7)$ and $(3, 5)$
        with slope $m=-1$.

        \part 

        In total, Alice can send Bob 12 packets, with one of the packets corrupted.
        For general errors, i.e., corruption, to protect against $k$ errors, a 
        sender must send $n+2k$ packages to protect against the errors. Therefore, 
        2 of the 12 packets sent by Alice are used to protect against one packet 
        corruption, and the largest message she can send Bob is $n=10$.

        % $P(x)$ is a degree 1 polynomial. $E(x)$ is a degree 2 polynomial since 
        % we now there have been two changes made. An $x$ where Bob is 
        % unable to uniquely determine $P(x)$ is one where the system of linear
        % equations $Q(i) = P(i)E(i)$ for $0 \leq i \leq 4$ does not have a single
        % solution for $Q(x)$. 

        % \begin{align*}
        %     &(i=0) & a_0 + 8b_0 \equiv 0 \pmod{13} \\
        %     &(i=1) & a_3 + a_2 + a_1 + a_0 + 6b_0 \equiv 7 \pmod{13} \\
        %     &(i=2) & a_3 + 4a_2 + 2a_1 + a_0 + (13-x)b_0\equiv 2x \pmod{13} \\
        %     &(i=3) & 6a_3 + 2a_2 + 3a_1 + a_0 + 8b_0 \equiv 2 \pmod{13} \\
        %     &(i=4) & a_3 + 2a_2 + 4a_1 + a_0 \equiv 0 \pmod{13}
        % \end{align*}

\end{homeworkProblem}
