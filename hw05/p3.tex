\begin{homeworkProblem}{To The Moon!}

    A secret number $s$ is required to launch a rocket, and Alice distributed 
    the values $(1,p(1)), (2,p(2)), \dots, 
    (n+1,p(n+1))$ of a degree $n$ polynomial $p(x)$ to a group of 
    \$GME holders $\text{Bob}_1, \dots, \text{Bob}_{n+1}$. As usual, she chose 
    $p(x)$ such that $p(0) = s$. $\text{Bob}_1$ through $\text{Bob}_{n+1}$ now 
    gather to jointly discover the secret. However, $\text{Bob}_1$ is secretly 
    a partner at Melvin Capital and already knows $s$, and wants to sabotage 
    $\text{Bob}_2,\dots, \text{Bob}_{n+1}$, making them believe that the secret 
    is in fact some fixed $s' \neq s$. How could he achieve this? In other 
    words, what value should he report (in terms variables known in the 
    problem, such as $s', s$ or $p(1)$) in order to make the others believe 
    that the secret is $s'$?

    \solution
    For a polynomial $p'(x)$ constructed by the holders using $\text{Bob}_1$'s
    $p'(1)$ (instead of his true $p(1)$), $p(x)$ and $p'(x)$ can be expressed as 
    follows from Lagrange Interpolation. Both have their first terms of the sum 
    seperated to highlight where $\text{Bob}_1$ fixes the result. 
    \[
        \begin{split}
            p(x) = p(1)\Delta_1(x)+ \sum_{i=2}^{n+1} p(i)\Delta_i(x) \\
            p'(x) = p'(1)\Delta_1(x)+ \sum_{i=2}^{n+1} p(i)\Delta_i(x)
        \end{split}
    \]
    For obtaining the secret, $p(x)$ is evaluated at $x=0$ to obtain $s$.
    Similarly, $p'(x)$ evaluated at $x=0$ should then equal $p'(0) = s'$. Although we 
    know that, for a basis polynomial $\Delta_i(x_j)$, when $i \not = j$, $\Delta_i(x_j)
    = 0$ and 1 otherwise for $i, j \in \{1, 2, \dots, n+1\}$, we are not sure what 
    $\Delta_i(x)$ evaluated at $x=0$ is out of the gate. 
    \[
        \begin{split}
            s = p(0) = p(1)\Delta_1(0)+ \sum_{i=2}^{n+1} p(i)\Delta_i(0) \\
            s' = p'(0) = p'(1)\Delta_1(0)+ \sum_{i=2}^{n+1} p(i)\Delta_i(0)    
        \end{split}    
    \]
    
    Solving for $p'(1)$ using the above equations for both the true and fixed secret
    results in a solution for $p'(1)$
    \[
        \begin{split}
            p'(1) = \frac{s' - \sum_{i=2}^{n+1} p(i)\Delta_i(0)}{\Delta_1(0)} \\
            p'(1) = \frac{s' - (s - p(1)\Delta_1(0))}{\Delta_1(0)} \\
            p'(1) = \frac{s' - s}{\Delta_1(0)} + p(1) \\
        \end{split}    
    \]
    From the defintion of the basis polynomials in Lagrange Interpolation,
    $\Delta_1(x) = \frac{\prod_{j=2}^{n+1} (x-x_j)}{\prod_{j=2}^{n+1} (1- x_j)}$,
    which for $x=0$ equals
    \[
        \Delta_1(0) = \frac{\prod_{j=2}^{n+1} (0-j)}{\prod_{j=2}^{n+1} (1 - j)} =
        \frac{\cancel{2} \cdot \cancel{3} \cdot \dots \cdot \cancel{n} \cdot (n+1)}
        {1 \cdot \cancel{2} \cdot \dots \cdot \cancel{n}} = 
        \frac{n+1}{1}
    \]
    Therefore, $p'(1) = \frac{s'-s}{n+1} + p(1)$.
    


    
    % He should report some value for $p(1)$ such that the reconstructed polynomial
    % from $(1, p(1)), (2, p(2)), \dots, (n+1, p(n+1))$, evaluated at 0, returns a
    % fixed $s'$ that is not the true secret. Reconstructing the polynomial with
    % Lagrange Interpolation, absent the fixing, the polynomial is of the form
    % \[
    %     p(x) = a_{n+1}x^{n+1} + a_{n}x^n + \dots + a_1 x_1 + a_0
    % \]
    % Therefore, $a_0$ must equal $s$, because $p(0) = a_{n+1} \cdot 0^{n+1} + \dots + 
    % a_1 \cdot 0 + a_0 = a_0$. $p(x)$ can be rewritten as
    % \[
    %     p(x) = \sum_{i=0}^{n+1} a_i x^i = \sum_{i=1}^{n+1} a_i x^i + s
    % \]
    % Without any fixing, $p(1) = \sum_{i=0}^{n+1} a_i \cdot 1^i = \sum_{i=1}^{n+1}
    % a_i + s$. In order for $\text{Bob}_1$ to make the others believe that the secret 
    % is $s'$, he must return a $p'(1)$ such that $p'(x)$ evaluated at $x=0$ returns 
    % $s'$. So, $p'(x)$ must be of a similar form to $p(x)$ but with a slight change
    % \[
    %     p'(x) = \sum_{i=0}^{n+1} a_i x^i = \sum_{i=1}^{n+1} a_i x^i + s'
    % \]
    % Evaluating $p'(x)$ at $x=1$ yields a similar form to $p(1)$, $p'(1) = \sum_{i=0}^{n+1} 
    % a_i \cdot 1^i = \sum_{i=1}^{n+1} a_i + s'$, which can be expressed as
    % \[
    %     p'(1) = \sum_{i=1}^{n+1} a_i \cdot 1^i + s' = \left( p(1) - s \right) + s'
    % \]
    % after substituting in $\sum_{i=1}^{n+1} a_i \cdot 1^i = p(1) - s$. 
    
    


\end{homeworkProblem}
