\begin{homeworkProblem}{Equivalent Polynomials}

    This problem is about polynomials with coefficients in $\text{GF}(p)$ for some 
    prime $p \in \mathbb{N}$. We say that two such polynomials $f$ and $g$ are 
    \emph{equivalent} if $f(x) \equiv g(x) \pmod{p}$ for every $x \in \text{GF}(p)$. 

    \begin{itemize}
        \item[A)] Show that $f(x)=x^{p-1}$ and $g(x)=1$ are \textbf{not} equivalent
        polynomials under $\text{GF}(p).$
        \item[B)] Use Fermat's Little Theorem to find a polynomial with degree 
        strictly less than 5 that is equivalent to $f(x) = x^5$ over $\text{GF}(5)$; 
        then find a polynomial with degree strictly less than 11 that is equivalent 
        to $g(x) = 4x^{70} + 9x^{11} + 3$ over $\text{GF}(11)$.
        \item[C)] In $\mathrm{GF}(p)$, prove that whenever $f(x)$ has degree $\ge 
        p$, it is equivalent to some polynomial $\tilde f(x)$ with degree $< p$.
    \end{itemize}

    \part 

    To show $f(x) \not \equiv g(x) \pmod{p}$, consider every $x \in \text{GF}(p)$.
    When $x=0$, $f(0) = 0^{p-1} = 0$ and $g(0) = 1$, which are not equivalent 
    under modulo $p$. Since $f(x)$ and $g(x)$ are not equivalent for all
    $x \in \text{GF}(p)$, they are not equivalent polynomials under $\text{GF}(p)$.
    
    \part 

    Using exponent rules in modular arithmetic in conjuction with Fermat's Little
    Theorem, $f(x) \equiv x^5 \equiv x^4 \cdot x \equiv x \pmod{5}$, and $x$ is a 
    polynomial with degree strictly less than 5. For the second part, $g(x) \equiv 
    4x^{70} + 9x^{11} + 3 \equiv 4(x^{10})^{7} + 9x^{10}x + 3 \equiv 9x + 7 \pmod{11}$,
    and $9x+7$ is a polynomial with degree strictly less than 11.
    
    \part 

    If $f(x)$ is a polynomial under $\text{GF}(p)$ with degree $\geq p$, then by
    recursively applying Fermat's Little Thereom to every term $x^k$, where $k \geq 
    p$, can be reduced to $x^k \equiv 1 \cdot x^r \pmod{p}$, since $x^k = x^{l(p-1) 
    + r}$ where $l, r \in \mathbb{Z}$. This results in a polynomial $\tilde{f}$ with
    degree $< p$ under in $\text{GF}(p)$ and is true for any $f(x)$ with degree $\geq 
    p$ over $\text{GF}(p)$.

    % If $f(x)$ has degree $\geq p$, it is possible, under $\text{GF}(p)$, to find
    % an equivalent polynomial since each exponent greater than $p$ can be expressed
    % in terms of $p-1$ and, using Fermat's Last Theorem, these terms are equivalent
    % to 1, only leaving the remaining terms with exponents less than $p$, resulting
    % in some polynomial with degree $< p$.

\end{homeworkProblem}